\begin{abstract}
Declarative style analysis has become one of widely used analysis techniques
for detecting potential bugs or security vulnerabilites.  Declarative style
analyzers are performed in three steps: creating database of facts out of
source code, evaluating rules for generating new facts, and performing query
over facts to extracting all information related to specific properties via
query system.  Declarative style analyzers broaden their analysis targets to
diverse programming languages. In order to target a new language, the
modification is made on first and second steps, but as the query system can be
independnet of language, the last step can be reused.  Altough the analyzers
target multiple programming languages, they are limited to support
single-languaged programs only and no know declarative style analyzer support
the analysis of multilingual programs written in two or more programming
languages.  With the growing use of mulitilingual prgrams in diverse areas, the
extended declarative anayzers that support the analysis of multillingual
analysis is highly demanded.

In this paper, we propose a systematic engineering methodology that easily
extends a declarative style analyzer supporting multiple target languages to
support multilingual program analysis as well. The main idea is to reuse the
existing components of original analyzer to minimize the burden of creating new
components. Our approach generates a merged database of facts that can be
separated into multiple logical language spaces so that existing
language-specific rules derive new facts from facts in the corresponding
language space, and defines language-interoperation rules that takes language
interoperation semantics into account.  Finally, the same query is performed to
get the analysis result considering language interoperation.  We develop a
proof-of-concept declarative style analyzer for multilingual programs by
extending CodeQL that can track dataflows across language boundaries for two
types of multilingual programs: Java-C programs written in Java and C
and Python-C programs written in Python and C.  We also implement a bug checker
on the top of the analyzer to detect dataflow related interoperation bugs in
real world Java-C programs.  In our evaluation, we show that our tool
successfully tracked dataflows across Java-C and Python-C language boundaries
and detected genuine interoperation bugs in real-world multilingual programs.
\end{abstract}

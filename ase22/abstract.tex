\begin{abstract}
Declarative style analysis has become one of widely used analysis techniques.
Declarative style analyzers consists of three parts: database facts, rules for
generating new facts, and query system for extracting all facts related to
specific properties.  Declarative style analyzers broaden their analysis
targets to diverse programming languages. The extension is on database and
rules are required for new language, but the query system does not require the
extension as it can be independnet of language. However, even if the analyzers
target multiple programming languages, it is limited to support they do not
directly support the analysis of multilingual programs written in two or more
programming languages.

In this paper, we propose a systematic engineering methodology that extends a
declarative style analyzer supporting multiple target languages to support
multilingual program analysis as well. Our approach generates a merged database
of facts that can be separated into multiple logical language spaces so that
existing language-specific rules derive new facts from facts in the
corresponding language space, then language-interoperation rules are defined so
that query system can extract facts considering language interoperation.  We
develop a proof-of-concept declarative style analyzer for multilingual programs
by extending CodeQL that can track data-flows across language boundaries for
two types of multilingual programs: Java-C programs and Python-C programs
written in both Python and C.  We also implement a bug checker on the top of
the analyzer to detect data-flow related interoperation bugs in real world
Java-C programs.  In our evaluation, we show that our tool successfully tracked
data-flows across Java-C and Python-C language boundaries and detected genuine
interoperation bugs in real-world multilingual programs.
\end{abstract}

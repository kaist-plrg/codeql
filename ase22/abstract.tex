\begin{abstract}
Declarative static program analysis has become one of the widely-used
program analysis techniques thanks to its logical specification:
it specifies ``what'' rather than ``how.''
Declarative static analyzers perform three steps: creating databases of facts from
program source code, evaluating rules to generate new facts, and running queries
over facts to extract all information related to specific properties via
query systems.
Declarative static analyzers can easily support diverse programming languages
as their analysis targets by extending only databases and rules
for new languages. Because query systems are independent of
programming languages, they are reusable for new languages.
However, even when declarative analyzers support multiple
programming languages, they do not support the analysis of multilingual
programs written in two or more programming languages.
With the growing prevalence of multilingual programs in diverse areas,
supporting multilingual program analysis is increasingly important.

In this paper, we propose a systematic methodology that extends a
declarative static analyzer supporting multiple target languages to support
multilingual programs as well. The main idea is to reuse
existing components of the analyzer to minimize the burden of creating new
components.  Our approach first generates a merged database
of facts, consisting of multiple logical language spaces. It allows
existing language-specific rules to derive new facts for the corresponding language
from the facts in the corresponding language space. Then, it defines
language-interoperation rules that handle the language  interoperation semantics.
Finally, it uses the same query system to get analysis results
leveraging the language interoperation semantics.
We develop a proof-of-concept declarative static analyzer for
multilingual programs by extending CodeQL, which can track dataflows
across language boundaries. The analyzer can analyze two types of multilingual programs:
JNI programs written in Java and C, and Python-C programs 
written in Python and C.  We also implement a bug checker on top of
the analyzer to detect dataflow-related interoperation bugs in real-world JNI programs.
Our evaluation shows that the analyzer successfully tracks
dataflows across Java-C and Python-C language boundaries and detects genuine
interoperation bugs in real-world multilingual programs.
\end{abstract}

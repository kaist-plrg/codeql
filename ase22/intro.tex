\section{Introduction}
With the success of Datalog[??], declarative style analyses have become one of
widely used analysis techniques. Datalog consists of three parts: a database of
facts, rules deriving new facts from existing facts, and a query system
extracting all facts related to specific properties by applying the rules to
the facts. Inspired by Datalog, declarative style analyzers detect possible
bugs and security vulnerabilities by transforming program source code to facts
and extracting new facts from the facts via a sequence of queries. DOOP[??] is
a declarative style analysis framework conducting points-to analysis for Java
programs, and CodeQL[??] is a declarative semantic code analysis engine
maintained by Github, which performs data-flow analysis to detect security
vulnerabilities for C++, C\#, and JavaScript programs. Recently, Glean[??], an
experimental declarative query system, has been developed by Meta, on which
users can query information about code structures for JavaScript and Hack
programs.  Declarative style analyzers broaden their analysis targets to
diverse programming languages. Because the query system is independent of
analysis target languages, only required extensions for an additional target
language support include a schema of the database containing facts,
language-specific rules, and a front-end transforming programs written in the
target language to facts. With the extensions, declarative analyzers can
analyze various properties in programs written in the new target language,
using a sequence of queries and the query system.  DOOP currently supports
Python program analysis to detect tensor shape mismatching bugs in TensorFlow
based Python ML models[??]. CodeQL can track data-flows not only in C++, C\#,
and JavaScript, but also in Java, Ruby, Python, and Swift programs.
Furthermore, Glean has a plan to support diverse target programming languages
including Python, Java, C++, Rust, and Haskell.

While the analyzers target multiple programming languages, they have limited
application to multilingual programs written in two or more programming
languages. Multilingual programs have been widely developed in various
application domains[5, 8]. However, multilingual programs are often vulnerable
to bugs or security issues more than monolingual programs. A large-scale study
on the code quality of multilingual programs[6] reported that using multiple
languages together correlates with higher error-proneness. Moreover, [7] showed
that two or three times more bugs and security issues had been reported on the
language interoperation than on the intraoperation in widely-used open-source
JNI programs such as OpenJ9 and VLC.

In this paper, we propose a systematic engineering methodology that extends a
declarative style analyzer supporting multiple target languages to support
multilingual program analysis as well. Our approach separates a database of
facts into multiple logical language spaces to put all facts of a multilingual
program into the database together by placing facts of different language parts
in different language spaces. Existing language-specific rules derive new facts
from facts in the corresponding language space. To handle language
interoperation in multilingual programs, we define language-interoperation
rules referring to language interoperation semantics, which derive new facts
from facts across language spaces. The extensions enable the query system to
extract facts in multilingual programs, considering language interoperation.

To evaluate the practicality of our approach, we develop a proof-of-concept
declarative style analyzer for multilingual programs by extending CodeQL. Our
analyzer tracks data-flows across language boundaries for two types of
multilingual programs: Java-C programs written in both Java and C, and Python-C
programs written in both Python and C. The extension is simple enough in that
it requires only a few lines of modifications of CodeQL and additional
language-interpretation rules. We also implement a bug checker on the top of
the analyzer to detect data-flow related interoperation bugs in Java-C and
Python-C programs. In our evaluation, we show that our tool successfully
tracked data-flows across Java-C and Python-C language boundaries and detected
genuine interoperation bugs in real-world multilingual programs.

The contributions of this paper are as followed:
\begin{itemize}
\item We propose an engineering method that extends a declarative style analyzer
supporting multiple target languages to support multilingual program analysis.

\item We implement a proof-of-concept declarative style data-flow analyzer for two
types of multilingual programs, Java-C and Python-C, by extending CodeQL.

\item We show the practical usefulness of our analyzer in the sense that it detects
data-flow related bugs at language boundaries of real-world multilingual
programs.
\end{itemize}

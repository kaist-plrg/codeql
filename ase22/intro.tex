\section{Introduction}
Declarative style analysis has become one of widely used analysis techniques.
Declarative style analyzers detect possible bugs and
security vulnerabilities by 1) transforming program source code to {\it database}
of {\it facts}, 2) generating new facts from the known facts via applying
{\it rules}, and 3) extracting information related to specific properties by
applying the {\it queries} to the facts via query system.  DOOP~\cite{doop} is a
declarative style analysis framework conducting points-to analysis for Java
programs, and CodeQL~\cite{codeql} is a declarative semantic code analysis engine
maintained by Github, which performs data-flow analysis to detect security
vulnerabilities in Java programs.  Recently, Glean~\cite{glean}, an experimental
declarative query system, has been developed by Meta, on which users can query
information about code structures for JavaScript and Hack programs.

Declarative style analyzers broaden their analysis targets to diverse
programming languages. In order to target a new language, each of three steps
of declarative style static analysis should be considered.  Because the query
system is independent of analysis target languages, the only required
modification is the first and the second step.  First, a new schema of the
database containing facts should be defined and a front-end program that
transforms the programs written in the target language into facts should be
implememted.  Second, new language-specific rules should be defined.  With the
modification, declarative analyzers can analyze the programs written in the new
target language, using a same queries and the query system.  DOOP currently
supports Python program analysis to detect tensor shape mismatching bugs in
TensorFlow based Python ML models~\cite{lagouvardos2020static}. CodeQL was
originally desinged for Java programs, and now it can track data-flows not only
in Java but also in C++, C\#, JavaScript, Ruby, and Python programs.
Furthermore, Glean has a plan to support diverse target programming languages
including Python, Java, C++, Rust, and Haskell.

While the analyzers target multiple programming languages, they do not directly
support the analysis of multilingual programs written in two or more
programming languages. Multilingual programs have been widely developed in
various application domains~\cite{kochhar2016large, mergendahlcross}. However,
multilingual programs are often vulnerable to bugs or security issues more than
monolingual programs. A large-scale study on the code quality of multilingual
programs~\cite{kochhar2016large} reported that using multiple languages
together correlates with higher error-proneness. Moreover,
\cite{grichi2020impact}~showed that two or three times more bugs and security
issues had been reported on the language interoperation than on the
intraoperation in widely-used open-source JNI programs such as OpenJ9 and VLC.

In this paper, we propose a systematic engineering methodology that extends a
declarative style analyzer supporting multiple target languages to support
multilingual program analysis as well. Our approach is to maximize the reuse of
already existing parts. First it generates a merged database of facts that can
be separated into multiple logical language spaces.  Each language space
consists of original facts from corresponding language database, and existing
language-specific rules derive new facts from facts in the corresponding
language space. Then, in order to handle language interoperation in
multilingual programs, we define language-interoperation rules referring to
language interoperation semantics. The language-interoperation rules derive new
facts from facts across language spaces. The extensions enable the query system
to extract facts in multilingual programs, considering language interoperation.
Finally, the same query is evaluated under same query system to get the
analysis result.

To evaluate the practicality of our approach, we develop a proof-of-concept
declarative style analyzer for multilingual programs by extending CodeQL. Our
analyzer tracks data-flows across language boundaries for two types of
multilingual programs: Java-C programs written in both Java and C, and Python-C
programs written in both Python and C. The extension is simple enough in that
it requires only a few lines of automated modifications of CodeQL and additional
language-interopretation rules. We also implement a bug checker on the top of
the analyzer to detect data-flow related interoperation bugs in real world Java-C programs.
In our evaluation, we show that our tool successfully tracked data-flows across
Java-C and Python-C language boundaries and detected genuine interoperation
bugs in real-world multilingual programs.

The contributions of this paper are as followed:
\begin{itemize}
\item We propose an engineering method that extends a declarative style analyzer
supporting multiple target languages to support multilingual program analysis.

\item We implement a proof-of-concept declarative style data-flow analyzer for two
types of multilingual programs, Java-C and Python-C, by extending CodeQL.

\item We show the practical usefulness of our analyzer in the sense that it detects
data-flow related bugs at language boundaries of real-world multilingual
programs.
\end{itemize}

\definecolor{codepurple}{rgb}{0.58,0,0.82}

\section{Overview}
In this section, we show an overview of how declarative style analysis
targetting multiple languages is performed with example of call graph
construction as a client analysis, and demonstrate our approach to extend the
analyzer to support analysis across languages.

\lstdefinestyle{mpython}{
    language=python,
    morekeywords={def},
    keywordstyle=\bfseries\color{blue!40!black},
    stringstyle=\color{red!40!black},
    basicstyle=\ttfamily\footnotesize,
    captionpos=b,
    numbers=left,
    numbersep=5pt,
    tabsize=2,
    xleftmargin=2.5em
}

\lstdefinestyle{mcpp}{
    keywordstyle=\bfseries\color{purple!40!black},
    language=c,
    stringstyle=\color{green!40!black},
    basicstyle=\ttfamily\footnotesize,
    captionpos=b,
    numbers=left,
    numbersep=5pt,
    tabsize=2,
    xleftmargin=2.5em
}

\begin{figure}[t]
  \centering
  \vspace{2mm}
  \begin{subfigure}[t]{0.5\textwidth}
    \begin{lstlisting}[style=mpython]
import f

def m1():
    f()
def m2():
    pass

m1()
    \end{lstlisting}
    \vspace*{-.5em}
    \caption{Code in language A}
    \label{fig:exam:langA}
  \end{subfigure}
  \begin{subfigure}[t]{0.5\textwidth}
    \begin{lstlisting}[style=mcpp,firstnumber=9]
void f() {
    CallMethod("m2");
}
    \end{lstlisting}
    \vspace*{-.5em}
    \caption{Code in language B}
    \label{fig:exam:langB}
  \end{subfigure}
  \vspace*{-.5em}
  \caption{Multilingual code example written in both language A and B}
  \label{fig:exam}
\end{figure}

Let us consider a multilingual program written in a Python-like language A
and a C-like language B.

The program in Figure~\ref{fig:exam} has two parts: Figure~\ref{fig:exam}(a)
shows a method \javacode{m1}, \javacode{m2} in language A, and
Figure~\ref{fig:exam}(b) shows a function \ccode{f} in language B.  The method
\javacode{m1} is called from top level of language A, and it calls the the
function \ccode{f} from language B.  Then, the function \ccode{f} calls back
the method \javacode{m2} in language A using API function, using the string
value "m2" to indicate the name of method.

\subsection{Analyses of single-language programs}
First, let's consider a hypothetical declarative style analyzer, that supports
language A.  As a first step, the database of facts are created from the source
code. THe facts created in this way may contain some syntactic information
about the source code. For example, we can think of two kinds of facts:
MethodAt(l, name) which indicates that the method name 'name' is defined at
line number l, and CallAt(l, name), which indicates there is a call to a method
named 'name' at line l.  Specifically, the program at Figure~\ref{fig:exam}(a)
would be transformed into the following facts:

MethodAt(3, "m1")

CallAt(4, "f")

MethodAt(5, "m2")

CallAt(8, "m1")

Next, there are rules that derice new facts from the known facts.  In this
example, we would want to define the rule for generating the facts of
CallEdge(l1, l2), which indicates there is an call edge from the call
expression at line l1 to the method definition at line l2. Using the know facts
of MethodAt and CallAt, we can define the rule like folowing:

CallEdge(l1, l2) :-

    CallAt(l1, call\_name),
    
    MethodAt(l2, meth\_name),
    
    call\_name = meth\_name.

The meaning of this rule is straight forward. Whenever there is a method
definition and a method call with the sane name, then there is c call edge
between them.  By applying this rule on top of facts obtained above, we would
get this new fact:

CallEdge(7, 3).

Finally, there is a query system that is used for extracting certain sets of
facts that meet certain condition. For example, we can simply make a query to
find every calledge using following query:

?- CallEdge(X, Y).

which should output the analysis result of (X, Y) = (7, 2).

This time, let's look at how the analyzer is extended to support the analysis
of B. Similar to language A, the first step is to create database. For language
B, we can have the rules of FunctionAt, CallAt, and Argument as follows:

FunctionAt(9, "f")

CallAt(10, "CallMethod")

Argument(10, 0, "m2")

where Argument(l, i, arg) would mean that the function call at line l would
have arg as the ith argument.  Then we can define the rule for CallEdge in
similar manner:

CallEdge(l1, l2) :-
    CallAt(l1, call\_name),
    FunctionAt(l2, func\_name),
    call\_name = func\_name.

It is important thing to note here is that, the same name of "CallEdge" is used
for this rule. The rationale for the developer of analyzers to use the same
name here, is that the exact same query can be reused. For example, we can use
the exact same query

?- CallEdge(X, Y).

to find all possible set of call edges. In this case, the output would be empty
as the analyzer is not considering the source code from A.

\subsection{Analysis of multilingual program}
So far, we have seen how a declarative analyzer designed for one language can
be extended to support another language. Now, we can take aadvantage of this

Again, the first step is to create merged database that can separated into 2
logical language spaces. In this example, the merged database would look like
this:

MethodAt\_A(3, "m1")

CallAt\_A(4, "f")

MethodAt\_A(5, "m2")

CallAt\_A(8, "m1")

FunctionAt\_B(9, "f")

CallAt\_B(10, "CallMethod")

Argument\_B(10, 0, "m2")

Then, the next step is to define the rule for CallEdge.  The new rule will be
composed of three parts: language-specific rule for A, lanuguage-specific rule
for B, and language-interoperations rules:

CallEdge(l1, l2) :-

    CallEdge\_A(l1, l2)
    
    ;
    
    CallEdge\_B(l1, l2)

    ;

    CallEdge\_AB(l1, l2)

    ;

    CallEdge\_BA(l1, l2).

For language-specific rules for CallEdge\_A and CallEdge\_B can be defined same
as before.  For language-interoperation rule, both call from A to B and B to A
should be considered. In our example language, call from A to B is similar to
normal method call from B to B, so the rule for CallEdge\_AB can be defined in
similar way:

CallEdge\_AB(l1, l2) :-

    CallAt\_A(l1, call\_name),
    
    FunctionAt\_B(l2, func\_name),
    
    call\_name = func\_name.

For the rule for CallEdge\_BA, the situation is a bit trickier.  Since the
method call is done via an api function call, we weill have to consider this
semantics.

CallEdge\_BA(l1, l2) :-

    CallAt\_B(l1, "CallMethod"),
    
    Argument\_B(l1, 0, arg),
    
    MethodAt\_A(l2, meth\_name),
    
    arg = meth\_name.

After fully defining the rule for CallEdge, then we can reuse the query system
just as before to findout every possible calledge:

?- CallEdge(X, Y).
which will give us the output, (X, Y) = (7, 3), (4, 9), (10, 5)

It is interesting to see what is the only required task for this extension.
Most of the parts in original analyzers is reused, and the only requirements
are the additional language-interoperation rules, in this case,
language-interoperation rules for CallEdge\_AB and CallEdge\_BA. Rest of the
part can be constructed in automated way.

Starting from the next section, we formally present our approach to extend the
analyzer targeting multiple lanugaes to support multilingual analysis, specific
implementation details on extedning CodeQL to support Java-C programs and
Python-C programs, and demonstrate our experimental result with the
proof-of-concept implementation.

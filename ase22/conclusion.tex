\section{Conclusion}\label{sec:conclude}
In this paper, we suggested the methodology of extending a declarative style
anlyzer that targets multiple languages into a analyzer that can analyze the
multilingual programs. First we create the database of fact that can be divided
into multiple logical language spaces, where each language space consists of
original facts. Then we define additional language-interoperation rules that
will generate the facts that reflect interoperation semantics between languages.
Finally, we perform the original query to obtain the analysis result.
Based on this approach, we build our prototype implementation \ours on top of
CodeQL.  Using \ours, we could sucessfully perform dataflow analysis on various
benchmark suites of JNI programs and extension module programs.  Also, we find
33 true bugs and vulnerabilities from real-world applications, 12 of which are
from applications that \lees failed to detect.  We believe that our approach is
applicable to analysis for other kinds of multilingual programs beyond Java-C
or Python-C, and to various types of analysi beyond dataflow analysis.

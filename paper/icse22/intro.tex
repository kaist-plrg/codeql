\section{Introduction}
Multilingual programs written in multiple programming languages
have been widely used in various application domains.
Since different languages have different benefits,
developers use multiple languages and their interoperations
via the Foreign Function Interface (FFI). For example, using the Java Native Interface (JNI),
JNI programs have high-level logic written in Java and performance-critical
modules implemented in C. By utilizing diverse languages, programmers can
develop complex code effectively, improve their program performance,
and reuse existing modules written in different programming languages.
\citet{kochhar2016large} reported that about a half of the most popular open-source
projects mainly written in one of the popular programming languages are
multilingual.

However, multilingual programs are often vulnerable to bugs or security issues
more than monolingual programs. Since the language interoperation semantics
via FFIs is often complex and different from the semantics of any
single language, programmers can easily make errors by mistake.
Unfortunately, such errors are difficult to catch and fix because existing
tools such as compilers, IDEs, and static analyzers cannot detect erroneous
behaviors involving the interoperation semantics.
A large-scale study on the code quality of multilingual programs~\cite{kochhar2016large}
reported that using multiple languages together correlates with higher error-proneness.  
Moreover, \citet{grichi2020impact} showed that two or three times more
bugs and security issues had been reported on the language
interoperation than on the intraoperation in widely-used open-source JNI
programs such as OpenJ9 and VLC.

For multilingual program analysis, researchers have proposed two approaches.
The first approach leverages general-purpose analyzers for monolingual
programs~\cite{JN-SAF, LeeASE20}.  It analyzes each module
written in a single language using its analyzer to generate
a summary as an abstraction of its behaviors.  Then, it performs a target
analysis to detect bugs and security vulnerabilities cross language boundaries
using the summaries. The second approach generates an abstraction
of each module written in each language in an IR that can
incorporate both languages, and performs
a target analysis using a general-purpose analyzer
running on the IR~\cite{hybridroid, sas2021}.
While both approaches can detect bugs and security vulnerabilities
in multilingual programs, they often fail to analyze large-scale real-world applications.
Because they utilize the full features of expensive general-purpose analyzers,
they may waste their analysis time on the features that are irrelevant to
the target analysis.

In this paper, we propose a novel approach that statically analyzes
multilingual programs using a declarative-style analysis~\cite{doop} for the first time.
A declarative-style analysis has been broadly used for monolingual programs,
and it consists of \textit{datafacts} and \textit{rules}.
A datafact is simply an ordered tuple of elements, which indicates the
information that the elements in the tuple have a certain relation.
A Rule defines how to derive a datafact from other datafacts.
A declarative-style analysis finds all the possible datafacts that can
be derived from given datafacts and rules.
Adopting this approach, we can statically analyze multilingual programs as follows.
The first step is to extract syntactic datafacts from the modules written in each language.
Then, the next step is to define intraoperation rules that
generate new datafacts using existing datafacts.
The final step is to implement a multilingual static analysis by defining
inter-language rules considering the interoperation semantics between different languages. 
Applying all the rules, including both intraoperation
and interoperation rules, produces the final analysis results in
the form of a set of datafacts.  The biggest advantage of a declarative-style analysis is its
\textit{query-based} analysis; a target analysis is specified in
special rules called queries. Only such datafacts that are
necessary to evaluate the query are computed.
Unlike all the previous approaches leverage general-purpose static analyzers,
which may lead to unnecessary computations, our approach does not spend time
on generating unnecessary summaries or computing irrelevant datafacts.


To show the usefulness of our approach, we implement a proof-of-concept static
analyzer for JNI programs written in both Java and C.
JNI programs have been actively developed for both desktop and mobile
applications. Still, they often contain diverse bad smells in code,
which may introduce bugs and security
issues~\cite{nishiwaki2012sean, grichi2019state, abidi2019behind, abidi2021multi}.
We implement our analyzer on top of CodeQL~\cite{codeql}, a
static analysis engine powered by a declarative language called QL.
It abstracts away the need to manage underlying data structures or implement
efficient algorithms for evaluating queries, thus enabling simple and fast query writing.
We evaluated the analyzer using real-world Android JNI applications
to see whether its performance is comparable to that of the existing work.
Our experimental results show that our analyzer outperforms the
state-of-the-art JNI analyzer, \lees~\cite{LeeASE20}, in terms of scalability.  Our analyzer
shows better scalability without analysis precision degradation.
In addition, we could easily develop various kinds of bug detectors
reported in the literature on top of our analyzer, thanks to the power of CodeQL.
Using these bug detectors, we found 33 true bugs and vulnerabilities from
real-world applications, among them 12 bugs are from the applications that \lees
failed to analyze.

The contributions of this paper include the following:
\begin{itemize}
  \item We propose a simple and effective declarative static analysis for
    multilingual programs for the first time. Our approach analyzes dataflows
    over language boundaries seamlessly without complex, expensive,
    and unnecessary computation.

\item We implement a prototype declarative-style JNI program analyzer on top of CodeQL.
  The design of our analyzer enables us to easily implement various bug
    detectors for JNI programs using the powerful query language
    features of CodeQL. 

\item We show the practical usefulness of the JNI program analyzer for real-world
  JNI programs. Our tool analyzed more applications than the
    state-of-the-art JNI program analyzer, \lees, and detected the same kinds of bugs
    and security vulnerabilities precisely.
\end{itemize}

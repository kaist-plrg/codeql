\section{Related Work}\label{sec:related}
For Java-C program analysis, ILEA~\cite{ILEA} extends
the Java Virtual Machine Language (JVML) and Jlint, a dataflow analyzer for
Java bytecode, to compile both Java and C code to the extended JVML and analyze
the integrated programs.  Since the modest extension of JVML cannot
support the full C semantics like the C memory model, it extremely over-approximates C
operations such as reads and writes through pointers.  Lee et al.~\cite{LeeASE20}
proposed a general approach to analyzing multilingual programs written in both
{\it host} and {\it guest} languages.  Their approach first uses a guest
language analyzer to extract semantics summaries from the parts written in the
guest language, C in their implementation, translates and integrates the
summaries into a host language, Java, and then performs the whole-program
analysis using a host language analyzer. 
%While their approach leverages the
%full features of existing static analyzers, \ours inherits the query-based
%analysis and can compute target properties effectively without expensive and
%redundant computation.

Researchers also studied binaries rather than C/C++ source code
for Java-C program analysis. Fourtounis et al.~\cite{scanning} proposed a lightweight reverse engineering technique to
recover Java method calls from binaries instead of performing heavy analyses on
binary code.  Their reverse engineering generates datafacts of Java method
calls from binaries, which can be used in further declarative-style Java
analyses. While the approach is lightweight but targets only Java method call
identification in binaries, our approach seamlessly analyzes dataflows across language
boundaries between Java and C.  JN-SAF~\cite{JN-SAF} defines a
unified dataflow summary to represent dataflows in both Java bytecode and
binary.  It extracts summaries from each Java method with a Java static
analyzer and from each native function with a binary symbolic execution, and
composes the summaries in a bottom-up manner to find data leakages over
language boundaries.
%It sacrifices the performance to analyze every possible
%dataflow from binaries using the expensive symbolic execution. On the contrary, our
%approach is scalable in that the query-based approach analyzes only necessary
%dataflows when C source code is available.

Android hybrid apps are
written in both Java and JavaScript, taking advantage of the portability from
JavaScript and device resource accessibility from Java.
HybriDroid~\cite{hybridroid}, implemented on top of WALA~\cite{WALA}, analyzes
Java and JavaScript code seamlessly to detect programmer errors on
interoperations and track data leakages across language boundaries.
Bae et al.~\cite{BaeICSE19} tackled the expensive analysis of HybriDroid and proposed
a lightweight type system detecting the same kinds of programmer errors in
Android hybrid apps. Jin et al.~\cite{jin2014code} proposed static detection
of code injection attacks from JavaScript to Java.  They manually
modeled Java frameworks and performed a taint analysis for JavaScript with the models.


Python supports an interoperability mechanism with C~\cite{PythonC}.
Developers often import performance-critical C code to high-level Python code.
Recent work~\cite{sas2021} proposed a Python-C analyzer by reusing existing
Python and C analyzers built on top of the same framework, MOPSA~\cite{Mopsa}.
They leverage the full features of the analyzers within the same framework to
perform precise context-sensitive value analyses. 
\inred{Polycruise~\cite{polycruise} is a Python-C analyzer that enables an efficeint
dynamic information flow analysis(DIFA).
(...)
In contrast to these works}, our work focuses on declarative style dataflow analysis. 

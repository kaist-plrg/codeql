\begin{figure}[t]
  \centering
  \vspace{2mm}
  \includegraphics[width=0.7\textwidth]{img/codeql.pdf}
  \caption{Overall structure of \ours for JNI programs}
  \label{fig:codeql}
\end{figure}

\section{\ours: Extension of CodeQL for multilingual program analysis}\label{sec:impl}
In this section, we present \ours, a prototype extension of
CodeQL~\cite{codeql} for multilingual program analysis.  \ours tracks dataflows
across language boundaries in two types of multilingual programs: 1) Java-C
programs interoperating via Java Native Interface(JNI)~\cite{jnispec} and 2)
Python-C programs interoperating via Python Extension Module~\cite{pyext}.  For
simplicity, we use the Java-C program analysis to explain the common parts and
emphasize the Python-C program analysis in Section~\ref{sec:merging2}.

\subsection{CodeQL}
CodeQL is a declarative static analysis engine that transforms source code into
a database of facts and performs analyses by evaluating queries written in the
declarative and object-oriented language called QL (Query Language). Using QL,
one can depict rules by defining \textit{ predicates} and \textit{ classes}. The following
QL code defines the \inblue{classical Datalog-style} unary predicate \codeql{isOneOrTwo}:

\begin{lstlisting}[style=codeql,xleftmargin=2.5em]
predicate isOneOrTwo(int n) {
  n = 1 or n = 2
}
\end{lstlisting}

\noindent
which states that the fact \codeql{isOneOrTwo(n)} is derivable from either
of the two facts, \codeql{n = 1} and \codeql{n = 2}.

\inblue{ QL allows syntactic definitions of predicates which, like C functions,
can have return types. These are desugared into classical predicates
by adding an additional argument named \codeql{result}.
For example, consider} the following predicate \codeql{addOne}:

\begin{lstlisting}[style=codeql,xleftmargin=2.5em]
int addOne(int n) {
  isOneOrTwo(n) and result = n + 1
}
\end{lstlisting}
\noindent
The predicate is a syntactic sugar equivalent to the following \inblue{classical} predicate:
\begin{lstlisting}[style=codeql,xleftmargin=2.5em]
predicate addOnePred(int n, int result) {
  isOneOrTwo(n) and result = n + 1
}
\end{lstlisting}
\noindent
\inblue{
Then, every use site of the original predicate can be rewritten to
use the desugared predicates.
For example, an equality formula \codeql{m = addOne(n)} is desugared to
to \codeql{addOnePred(n, m)}.

One charateristic feature of QL is that it supports syntax of class similar to object-oriented
programing languages like Java. A QL class should be extended from another type,
and should define a special predicate inside the class, called {\it
characteristic predicate}. The characteristic predicates
resemble the syntax of the traditional constructors in other object-oriented languages;
it has the same name to the class and does not have return type.
Also, a QL class allows defining member predicates, as if defineing the methods of class.
These kinds of syntax allows the user to write intuitive QL codes in objected-oriented manner.
}
The following QL code defines a class named \codeql{OneOrTwo} \inblue{that extends \codeql{int}
with the characterisct predicate and one member predicate.}
:
\begin{lstlisting}[style=codeql,xleftmargin=2.5em]
class OneOrTwo extends int {
  // characteristic predicate
  OneOrTwo() { this = 1 or this = 2 }
  // member predicate
  int add(OneOrTwo that) { result = this + that }
}
\end{lstlisting}
\noindent

\inblue{
Despite its syntax, QL class is also a syntactic sugar of classical
predicates.  For example, the characterstic predicate \codeql{OneOrTwo} can be
desugared into a unary predicate, \codeql{isOneOrTwo} Note that the type of the
argument \codeql{this} is \codeql{int}, which the class \codeql{OneOrTwo} is
extending.
}
\begin{lstlisting}[style=codeql,xleftmargin=2.5em]
predicate isOneOrTwo(int this) {
  this = 1 or this = 2
}
\end{lstlisting}
\inblue{
\noindent
Similarly, The member predicate \codeql{add} can be desugared into a ternary predicate
by introducing two more arguments, \codeql{this} and \codeql{result}.
}
\begin{lstlisting}[style=codeql,xleftmargin=2.5em]
predicate addPred(OneOrTwo this, OneOrTwo that, int result) {
  result = this + that
}
\end{lstlisting}
\inblue{
Then, every use site of the class can also be transforemd to use the desugared predicates.
For example, \codeql{z = x.add(y)} where x has a type of class \codeql{OneOrTwo}
can be desugared into \codeql{isOneOrTwo(x) and addPred(x, y, z)}.
}
For more detailed information about CodeQL \inblue{and its language QL}, refer to the paper of Avgustinov et
al.\cite{ql2016} or the official document~\cite{codeql}.

Figure~\ref{fig:codeql} presents the overall structure of \ours for JNI
programs.  First, it generates databases for both languages, C\footnote{ Even
though \ours analyzes JNI programs written in Java and both C and C++, this
paper refers to C only for presentation brevity.} and Java, and merges them
into one database.  This corresponds to the step of extracting the initial
syntactic facts from the source code.  Then, \ours~merges the common rules in
both languages, which are parts of their libraries that CodeQL provides, into
one merged library.  Finally, using the merged database and merged library, a
user can write a query to perform a client-analysis and evaluate it to produce
its analysis result.

\subsection{Creating Databases}
For compiled languages such as C and Java, CodeQL generates their databases by
compiling source programs.  When a compiler compiles a program, CodeQL monitors
the compiler to extract necessary information and creates a database with the
extracted information. For scripting languages like Python, CodeQL uses its own
extractor to directly extract necessary information from source code.  CodeQL
creates a database for a single language in two steps: 1) it stores the
extracted information in a \inblue{single} \textit{trap} file, a human-readable format file for
the CodeQL database, and 2) it transforms the trap file into a database in
binary format.  For example, the following demonstrates a sample trap file:

\begin{lstlisting}[style=java,numbers=none]
#10001=@"class;myClass.MyClass"
#10002=@"type;int"
primitives(#10002,"int")
#10003=@"callable;{#10001}.myMethod({#10002}){#10002}"
#10004=@"params;{#10003};0"
params(#10004,#10002,0,#10003,#10004)
paramName(#10004,"myParam")
...
\end{lstlisting}

\noindent
which describes the parameter information of the method \javacode{myMethod}.
It also shows facts that will eventually be stored in tables:
\javacode{primitives(...)}, \javacode{params(...)}, and
\javacode{paramName(...)}.  The database manages tables based on fact types.
For example, it stores the fact \javacode{primitives(...)} in a table named
\javacode{@primitives}, and \javacode{params(...)} in a different table named
\javacode{@params}.

To create a single merged database for both C and Java, \ours maintains a
separate trap file for each language and then merges them into a trap file.
One problem is that if each trap file has a table with the same name, the merge
fails due to the name collision.  To avoid the problem, \ours renames each
table to a globally unique name by appending a language-specific prefix to the
table name before the merge.  For example, if a table named \codeql{@params}
exists in both C and Java trap files, \ours renames the table in C to
\codeql{@c\_params} and the table in Java to \codeql{@java\_params}.  After
renaming such tables, \ours merges the trap files into a database on which
predicates and queries are evaluated.


\subsection{Lifting Libraries}

CodeQL provides various libraries for C and Java, including pre-defined
predicates and classes for users to implement their own analyses.  A dataflow
analysis library is such a library, which supports both C and Java.  For
example, Figure~\ref{fig:qll} (a) and (b) show sample QL libraries for C and
Java, respectively, which use the same class name \ccode{Node} and the same
predicate name \ccode{localFlowStep}.  However, even with the same name, the
class \ccode{Node} in C and the class \javacode{Node} in Java are different
classes, which are incompatible.  The same applies to the predicate
\ccode{localFlowStep}.  In other words, we can not use the class \ccode{Node}
in C as an argument of the predicate \javacode{localFlowStep} in Java or vice
versa.

\begin{figure}[hbt!]
  \centering
%  \vspace{2mm}
  \begin{subfigure}{0.94\textwidth}
\begin{lstlisting}[style=codeql,xleftmargin=2.5em]
class Node { ... }

predicate localFlowStep(Node from, Node to) {
  // Expr -> Expr
  exprToExprStep_nocfg(from.asExpr(), to.asExpr())
  or
  // Assignment -> LValue post-update node
  ...
}
\end{lstlisting}
    \vspace*{-.5em}
    \caption{\normalsize c/dataflow/internal/DataFlowUtil.qll}
    \label{fig:qll1}
  \end{subfigure}
  \begin{subfigure}{0.94\textwidth}
\begin{lstlisting}[style=codeql,xleftmargin=2.5em]
class Node { ... }

predicate localFlowStep(Node node1, Node node2) {
  // Variable flow steps through assignment expression
  node2.asExpr().(AssignExpr).getSource() = node1.asExpr()
  or
  // Variable flow steps through adjacent def-use and use-use pairs.
  ...
}
\end{lstlisting}
    \vspace*{-.5em}
    \caption{\normalsize java/dataflow/internal/DataFlowUtil.qll}
    \label{fig:qll2}
  \end{subfigure}
  \begin{subfigure}{0.94\textwidth}
\begin{lstlisting}[style=codeql,xleftmargin=2.5em]
module C { /* original classes and predicates from C lib */ }
module JAVA { /* original classes and predicates from Java lib */ }

private newtype TNode =
  TJavaNode(JAVA::Node n)
  or
  TCNode(C::Node n)
class Node extends TNode {
  JAVA::Node asJavaNode() { this = TJavaNode(result)}
  C::Node asCNode() { this = TCNode(result) } ...
}

predicate localFlowStep(Node node1, Node node2) {
  JAVA::localFlowStep(node1.asJavaNode(), node2.asJavaNode())
  or
  C::localFlowStep(node1.asCNode(), node2.asCNode())
}
\end{lstlisting}
    \vspace*{-.5em}
    \caption{\normalsize jni/dataflow/internal/DataFlowUtil.qll}
    \label{fig:qll3}
  \end{subfigure}
  \vspace*{-.5em}
  \caption{QL libraries for C and Java}
  \label{fig:qll}
\end{figure}

To make classes and predicates in C and Java compatible, \ours lifts each
library to the common level.  First, as on lines 1 and 2 of
Figure~\ref{fig:qll3}, \ours encapsulates the original dataflow libraries for C
and Java in separate modules named {\tt C} and {\tt Java},
respectively\footnote{https://codeql.github.com/docs/ql-language-reference/modules/}.
After the encapsulation, the original classes and predicates can be referenced
via the module containing them.  For example, \codeql{Java::Node} refers to the
original class \codeql{Node} in the module {\tt Java}.  Lines 4 to 11 of
Figure~\ref{fig:qll3} demonstrate the lifted class of the original class {\tt
Node}.  \ours lifts a class by first defining a sum
type\footnote{https://codeql.github.com/docs/ql-language-reference/types/\#algebraic-datatypes},
denoting that the lifted class comes either from C or Java, and then making the
lifted class of that type.  The lifted class defines two member predicates,
\codeql{asCNode} and \codeql{asJavaNode}, that downcast the elements of the
lifted class into the elements of the original C or Java class.  Lines 13 to 17
of Figure~\ref{fig:qll3} demonstrate the lifted predicate of the original
predicate {\tt localFlowStep}.  Similarly, \ours lifts a predicate by combining
two original predicates with the \codeql{or} connective. For each predicate,
arguments and return values are downcasted to the elements of the class of
their corresponding language.  After lifting, lifted predicates show the
equivalent behaviors as the original ones if all the arguments are from the
same language.

\ours can fully automate the lifting process using QL compiler error messages.
When compiling a query without importing the library, the QL compiler reports
error messages containing required classes and predicates, and signatures of
the predicates. Using the information, \ours automatically synthesizes lifted
classes and predicates.

\subsection{Merging Libraries: Java-C}\label{sec:merging}
After lifting libraries for different languages, we manually extend predicates
to reflect the interoperation semantics between multiple languages.  For the
Java-C program analysis, we identified various interactions from Java to C and
vice versa via JNI, and extended predicates to model their behaviors.  For
example, the following shows how we extend the \inblue{CodeQL} predicate
named \codeql{viableCallable}:
\begin{lstlisting}[style=codeql,xleftmargin=2.5em]
DataFlowCallable viableCallable(DataFlowCall c) {
  result.asJavaNode()    = JAVA::viableCallable(c.asJavaNode())
  or result.asCNode()    = C::viableCallable(c.asCNode())
  or result.asCNode()    = viableCallableJ2C(c.asJavaNode())
  or result.asJavaNode() = viableCallableC2J(c.asCNode())
}
\end{lstlisting}

\noindent
It finds call edges from call expressions to their targets.  Lines 2 and 3 show
the results of lifting using original predicates from the dataflow libraries.
They handle intra-language call edges from Java to Java and from C to C,
respectively.  Lines 4 and 5 show the results of merging libraries,
representing inter-language call edges.  The predicate
\codeql{viableCallableJ2C} finds call edges from Java to C, and
\codeql{viableCallableC2J} finds call edges from C to Java.

\textbf{Java to C.} In Java-C programs, one can make interactions from Java to
C by calling native functions in C from Java code.  The target of such a
function call is determined in a static manner.  The target function should
follow the JNI naming convention, which is adding \codeql{Java\_} as prefix,
followed by a fully qualified name of its class and the additional \codeql{\_},
to the method name.  For example, the target function name for a function call
of \codeql{cfunction} would be
\codeql{Java\_fully\_qualified\_class\_name\_cfunction}.  With this convention,
we can define \codeql{viableCallableJ2C} so that \codeql{f =
viableCallableJ2C(call)} holds when \codeql{f.toString() = "Java\_" +
call.getTarget().className() + "\_" +} \\ \codeql{call.getTarget().getName()}
holds.

\textbf{C to Java.} The interaction from C to Java is more complex and requires
more careful implementation than the interaction from Java to C.  The primary
difference is that a method call from C to Java requires the runtime values of
variables, which may not be always possible.  First, C code calls the interface
function \ccode{GetMethodID(name, sig)} to get the ``method ID'' of the Java
method whose name matches the first argument and the type signature matches the
second argument passed to this function.  This method ID is stored at a
variable, say \ccode{mid}, and an actual method call is invoked by another
interface function, \ccode{Call<type>Method(obj, mid, args...)}. Calling this
interface function corresponds to calling the method that \ccode{mid} indicates
with \ccode{obj} as ``this object'' and \ccode{args} as the arguments.

To correctly handle this method call, we should be able to answer these
questions: ``When we call \ccode{GetMethodID}, what are the string values of
\ccode{name} and \ccode{sig}?'' and ``When we call \ccode{Call<type>Method},
what is the method ID value of \ccode{mid}?'' Soundly answering these questions
requires inter-language dataflow analysis because the string or method ID
values may be passed across language boundaries.  However, we observed that
such a pattern rarely happens in practice, and using only intra-language
dataflow analysis within C code is enough for most cases.  Therefore, we
decided to sacrifice the \inblue{soundness} by using intra-language analysis instead of
inter-language analysis.
\inblue{
This may result in missing some call edges in rare cases,
}
as a trade-off for a more lightweight and simpler
implementation.  We implemented two intra-language flow analysis modules for C,
which find 1) dataflows from string literals to the arguments of interface
functions and 2) dataflows from interface function call results to the
arguments of interface functions.  Using these modules, we can implement the
predicate \codeql{viableCallableC2J} by adding a call edge from a
\ccode{Call<type>Method} call to the method \ccode{m}, if there is a flow to
\ccode{mid} from a call to \ccode{GetMethodID}, and string values that flow
into the arguments \ccode{name} and \ccode{sig} of \ccode{GetMethodID} that
correspond to the name and the type signature of the method \ccode{m}.

In addition to the predicate \codeql{viableCallable}, we extended more
predicates to consider other JNI interface functions such as \ccode{findClass}
and \ccode{GetFieldID}.  Most of such extended predicates are specialized
\codeql{step} predicates.  We extended them in a similar way to the calls from
C to Java described above.

\subsection{Merging Libraries: Python-C}\label{sec:merging2}
To analyze Python-C programs, we extended predicates to model the
interoperation semantics of Python Extension Module.  The following shows
\codeql{viableCallable} predicate we extended for the Python-C program
analysis:

\begin{lstlisting}[style=codeql,xleftmargin=-.5em,numbers=none]
DataFlowCallable viableCallable(DataFlowCall c) {
  result.asPythonNode()    = PYTHON::viableCallable(c.asPythonNode())
  or result.asCNode()      = C::viableCallable(c.asCNode())
  or result.asCNode()      = viableCallableP2C(c.asPythonNode())
  or result.asPythonNode() = viableCallableC2P(c.asCNode())
}
\end{lstlisting}

\noindent
The only different parts from the Java-C program analysis are the predicates
\codeql{viableCallableP2C} and \codeql{viableCallableC2P}.

\textbf{Python to C.} Similar to the interactions from Java to C, one can make
interactions from Python to C by importing and calling functions from C. Python
Extension Module provides a pre-defined C struct \ccode{PyMethodDef} to export
a C function to Python:

\begin{lstlisting}[style=mcpp]
struct PyMethodDef methods[] = {
  {
    .ml_name = "cfunction",
    .ml_meth = cfunction_impl, ...
  }, ...
}
\end{lstlisting}

\noindent
The member field \ccode{ml_name} has a visible name of a C function to Python,
and the member field \ccode{ml_meth} has the pointer to the actual C function.
Thus, the function \ccode{cfunction_impl} defined in C code is invoked when
importing and calling \ccode{cfunction} in Python code.  To model the behavior,
we define the CodeQL class \codeql{PyMethodDef} and the predicate
\codeql{viableCallableP2C}. The class \codeql{PyMethodDef} corresponds to the C
structure \ccode{PyMethodDef}. It has two member predicates \codeql{getName()}
and \codeql{getFunc()} whose results are the values of the fields
\ccode{ml_name} and \ccode{ml_meth}, respectively.  Then, we make a rule for
\codeql{viableCallableP2C} such that for some \codeql{def} of type
\codeql{PyMethodDef}, if \codeql{def.getName() } \codeql{=
call.getTarget().toString()} for some \codeql{call}, then \codeql{def.getFunc()
} \codeql{= viableCallableP2C(call)} holds.

In addition, we define rules that connect dataflows from arguments of a
Python-to-C function call to parameters of its target C function.  Unlike the
Java-to-C function call semantics, Python Extension Module packs arguments in a
Python tuple object and propagates the object to a single parameter of the
target C function.  Then, the target C function unpacks the tuple object into
individual Python objects by calling the \ccode{PyArg_ParseTuple} API.  This
means that the usual way of connecting arguments and parameters via their
positions does not work, and we need another way.  We modelled this behavior by
defining \codeql{VirtualArgNode}, a subclass of \codeql{Node}, that corresponds
to the Python tuple object. Then, we define some predicates for representing
flows in and out of this node. First, we define a step that represents the flow
from values of the original argument nodes to virtual argument nodes. We then
define a step that represents the flow from the values of virtual argument
nodes to a single parameter node in a C function.  By doing so, the flows from
original argument nodes to use sites will be found in three steps: 1) from an
argument node to a virtual argument node, 2) from the virtual argument node to
the parameter node of a C function, and 3) from the parameter node to the out
node of \ccode{PyArg_ParseTuple} API function call.

\medskip

\textbf{C to Python.} The interaction from C to Python requires the runtime
values of variables, similar to the interaction from C to Java.  To invoke
Python functions in C, C code calls the interface function
\ccode{PyObject_CallObject(func, args)}, where \ccode{func} is a Python
function object, and \ccode{args} is a Python tuple object that contains all
arguments.  Because C code can get Python objects as function arguments of the
Python-to-C function calls, we define two ``intra-language flows'' analyses to
identify Python objects assigned to \ccode{func}: 1) dataflows from a Python
function object to the argument of a C function call, and 2) dataflows from the
parameters of a C function to the first argument of
\ccode{PyObject_CallObject}.  Using these intra-language flow analyses, we can
find actual targets that should be invoked for \ccode{PyObject_CallObject} API
function calls.

\definecolor{codepurple}{rgb}{0.58,0,0.82}

\section{Declarative Static Analysis for Multilingual Programs}
This section outlines our approach to extending an existing declarative static
analysis for multilingual program analysis.  We show how an example declarative
static analysis works for call graph construction in two different monolingual
programs, and how we extend the analysis to construct a unified call graph
encompassing language boundaries.

\begin{figure}[t]
  \centering
  \vspace{2mm}
  \includegraphics[width=0.9\textwidth]{img/ov2.pdf}
  \caption{Declarative analysis for multilingual programs}
  \label{fig:ov2}
%\vspace*{-.5em}
\end{figure}

\subsection{Overview}
Figure~\ref{fig:ov2} illustrates how we support multilingual analysis in a
declarative style in the case of two languages as an example. The declarative
analyzer now gets two sets of syntactic facts extracted from two different
languages. In addition, new language-interoperation rules are defined on top of
the original language-specific rules from two languages to take the
interoperation semantics into account. Then, the same query is performed to get
analysis results.

\subsection{Example Declarative Static Analysis}
An example declarative static analysis consists of the following Datalog-like facts, rules,
and queries:

\[
  \begin{array}{llll}
    f & ::= & p(\overline{k\ |\ x}) & \text{\sc Fact}\\
    r & ::= & f'\ \mbox{:-}\ \overline{f\ |\neg f} & \text{\sc Rule}\\
    q & ::= & \mbox{?-}\ \overline{f}\ &  \text{\sc Query}
\end{array}
\]

\noindent
We use the overline notation to indicate 0 or more repetitions.  For example,
$\overline{x}$ is a shorthand for $x_1, x_2, ..., x_n$ for some $n \ge 0$.  The
fact $p(\overline{k\ |\ x})$ denotes a relation $p$ between constants
$\overline{k}$, such as string and integer literals, or variables
$\overline{x}$.  A rule $f'\ \mbox{:-}\ \overline{f\ |\neg f}$ consists of two
parts: the head of the rule $f'$ and the body of the rule $\overline{f\ |\neg
f}$, where the optional prefix $\neg$ denotes the negation.  \footnote{A
syntactic restriction exists for negation and recursive rules: a fact should
not be derived from its negation. For example, the rule $R(x, y)\ \mbox{:-}\
\neg R(x, y)$ is not a syntactically valid rule since $R(x, y)$ cannot be
derived from $\neg R(x, y)$.} It denotes that the head is derivable from the
body: $f'$ is derivable if all the facts $f'' \in \{\overline{f}\}$ belong to
known facts, and all the facts $f'' \in \{\overline{\neg f}\}$ do not belong to
known facts. \inblue{Note that a derived fact should not contain
any free variables as its argument.}
The body of the rule can be empty; in that case, the head of the
rule is vacuously added to known facts.  Such bodiless rules are the initial
input facts for the analysis.  The query $\mbox{?-}\ \overline{f}$ finds all
possible variable assignments that make all facts in $\overline{f}$ belong to
known facts if replacing variables in $\overline{f}$ with their corresponding
constants.


\subsection{Declarative Call Graph Construction for Monolingual Programs}\label{lab:ovmono}
\begin{figure}[t]
  \centering
  \begin{subfigure}[t]{0.45\textwidth}
    \begin{lstlisting}[style=mpython]
import f
def m1():
    f()
def m2():
    return None
m1()
    \end{lstlisting}
    \vspace*{-.5em}
    \caption{Code in Python-like language A}
    \label{fig:exam:langA}
  \end{subfigure}
  \begin{subfigure}[t]{0.45\textwidth}
    \begin{lstlisting}[style=mcpp,firstnumber=7]
void f() {
    CallMethod("m2");
}
    \end{lstlisting}
    \vspace*{3.5em}
    \caption{Code in C-like language B}
    \label{fig:exam:langB}
  \end{subfigure}
  \vspace*{-.5em}
  \caption{Multilingual code example written in both A and B}
  \label{fig:exam}
\end{figure}

Figure~\ref{fig:exam} shows a multilingual code example written in (a)
Python-like language {\it A} and (b) C-like language {\it B}. In {\it A} code,
there are two functions, \rcode{m1} on line~2 and \rcode{m2} on line~4. The
function \rcode{ m1} calls the function \rcode{f} in {\it B} code via language
interoperation between {\it A} and {\it B}, and the function \rcode{m2} does
nothing. On line~6 in {\it A}, the expression \rcode{m1()} calls the function
\rcode{m1} defined on line~2 to invoke the function \rcode{f} in {\it B} code.
The function \rcode{f} in {\it B} code calls the \rcode{CallMethod} function
with a string literal \rcode{"m2"} to invoke the function \rcode{m2} in {\it A}
code, using the string literal argument as the target function name.


A declarative static analysis first creates a database of facts for the code.
Initial facts are just syntactic information extracted from the code.  For
example, there are two kinds of facts required for call graph construction:
\rcode{FunctionAt(lineNum, name)} denotes function definition information
containing its line number \rcode{lineNum} and the function name \rcode{ name},
and \rcode{CallAt(lineNum, name)} denotes callsite information containing the
callsite line number \rcode{lineNum} and its target function name \rcode{
name}. The initial facts for the code in Figure~\ref{fig:exam:langA} are as
follows:

\begin{lstlisting}[style=mrule]
    FunctionAt(2, "m1")
    CallAt(3, "f")
    FunctionAt(4, "m2")
    CallAt(6, "m1")
\end{lstlisting}

Then, the analysis requires rules for the initial facts to derive new facts
about function call relations. In this example, one kind of facts, \rcode{
CallEdge(lineNumCall, lineNumFunc)}, denotes a call relation from a callsite on
line \rcode{ lineNumCall} to a function definition on line \rcode{
lineNumFunc}. To derive function call relation facts, we define a rule as
follows:

\begin{lstlisting}[style=mrule]
    CallEdge(lineNumCall, lineNumFunc) :-
      CallAt(lineNumCall, name),
      FunctionAt(lineNumFunc, name)
\end{lstlisting}

\noindent
The rule derives a new fact \rcode{ CallEdge(lineNumCall, lineNumFunc)} when a
callsite on line \rcode{ lineNumCall} has the same target function name as the
name of a function defined on line \rcode{ lineNumFunc}. Therefore, a new fact
can be derived from the initial facts as follows:

\begin{lstlisting}[style=mrule]
    CallEdge(6, 2)
\end{lstlisting}

Finally, a query system extracts a set of function call relation facts when we
make a query.  For example, the following query finds all possible literals for
the variables \rcode{ X} and \rcode{ Y}:

\begin{lstlisting}[style=mrule]
    ?- CallEdge(X, Y).
\end{lstlisting}
\noindent
Since only one \rcode{ CallEdge} fact, \rcode{ CallEdge(6,2)}, belongs to known
facts, the query system produces \rcode{ (X, Y)\ $\in$ \{(6, 2)\}} as a result
of the query.

Let us consider the language {\it B} case.  The initial facts for the code in
Figure~\ref{fig:exam:langB} are as follows:

\begin{lstlisting}[style=mrule]
    ProcedureAt(7, "f")
    CallAt(8, "CallMethod")
    Argument(8, 0, "m2")
\end{lstlisting}

\noindent
The fact \rcode{ ProcedureAt(lineNum, name)} denotes that a procedure of name
\rcode{name} is defined on line \rcode{lineNum} similarly to
\rcode{FunctionAt(lineNum, name)} for the language {\it A}.  The fact \rcode{
Argument(lineNum, i, arg)} denotes that the value of the \rcode{i}-th argument
of a function call on line \rcode{ lineNum} is \rcode{ arg}.

Then, we define a slightly different rule to derive a function call relation
from the initial facts as follows:

\begin{lstlisting}[style=mrule]
    CallEdge(lineNumCall, lineNumFunc) :-
      CallAt(lineNumCall, name),
      ProcedureAt(lineNumFunc, name)
\end{lstlisting}

\noindent
This example shows a language-specific rule: the same fact \rcode{CallEdge} can
be derived from different facts for different languages. When we make the query
\rcode{ ?- CallEdge(X, Y)}, the query system produces \rcode{ (X, Y)\ $\in$
\{\}} as a result since it has no facts to satisfy the rule.

\subsection{Extension of Declarative Call Graph Construction for Multilingual
Programs} The first step of our extension for multilingual program analysis is
to create a merged database that consists of two logical language spaces, {\it
A} and {\it B}. Then, we store the initial facts of the code in
Figure~\ref{fig:exam:langA} and Figure~\ref{fig:exam:langB} to their
corresponding language spaces, respectively. The following shows the merged
database containing the initial facts, where the subscripts of the facts denote
the logical language spaces in which the facts reside:

\begin{lstlisting}[style=mrule]
    FunctionAt$_A$(2, "m1")
    CallAt$_A$(3, "f")
    FunctionAt$_A$(4, "m2")
    CallAt$_A$(6, "m1")
    ProcedureAt$_B$(7, "f")
    CallAt$_B$(8, "CallMethod")
    Argument$_B$(8, 0, "m2")
\end{lstlisting}

The next step defines generalized rules to derive function call relation facts.
We define them by composing language-specific rules for A, language-specific
rules for B, and language-interoperation rules as follows:

\begin{lstlisting}[style=mrule]
    CallEdge(lineNumCall, lineNumFunc) :- CallEdge$_A$(lineNumCall, lineNumFunc)
    CallEdge(lineNumCall, lineNumFunc) :- CallEdge$_B$(lineNumCall, lineNumFunc)
    CallEdge(lineNumCall, lineNumFunc) :- CallEdge$_{AB}$(lineNumCall, lineNumFunc)
    CallEdge(lineNumCall, lineNumFunc) :- CallEdge$_{BA}$(lineNumCall, lineNumFunc)
\end{lstlisting}

\noindent
We slightly modify the language-specific rules for {\it A} and {\it B} to
derive the \rcode{ CallEdge$_A$} and \rcode{ CallEdge$_B$}:

\begin{tabular}{ll}
  {\begin{lstlisting}[style=mrule]
    CallEdge$_A$(lineNumCall, lineNumFunc) :-
      CallAt$_A$(lineNumCall, name),
      FunctionAt$_A$(lineNumFunc, name)
  \end{lstlisting}} &
  {\begin{lstlisting}[style=mrule]
    CallEdge$_B$(lineNumCall, lineNumFunc) :-
      CallAt$_B$(lineNumCall, name),
      ProcedureAt$_B$(lineNumFunc, name)
  \end{lstlisting}}
\end{tabular}

\noindent
In addition, we define additional language-interoperation rules to derive
\rcode{ CallEdge$_{AB}$} and \rcode{ CallEdge$_{BA}$}. The fact \rcode{
CallEdge$_{AB}$} denotes a function call relation from a function written in
the language {\it A} to a function written in the language {\it B}, and \rcode{
CallEdge$_{BA}$} denotes a function call relation in the opposite direction.
Since the function call semantics from {\it A} to {\it B} is the same as the
normal function call semantics in {\it A}, we define the interoperation rule
from {\it A} to {\it B} as follows:

\begin{lstlisting}[style=mrule]
    CallEdge$_{AB}$(lineNumCall, lineNumFunc) :-
      CallAt$_A$(lineNumCall, name),
      ProcedureAt$_B$(lineNumFunc, name)
\end{lstlisting}

\noindent
On the other hand, the function call semantics from {\it B} to {\it A} is
different from the normal function call semantics in {\it B}. The language {\it
B} calls a function written in {\it A} by calling an interoperation API
function \rcode{ CallMethod} with a target function name as the first argument.
Thus, we define the interoperation rule from {\it B} to {\it A} as follows:

\begin{lstlisting}[style=mrule]
    CallEdge$_{BA}$(lineNumCall, lineNumFunc) :-
      CallAt$_B$(lineNumCall, "CallMethod"),
      Argument$_B$(lineNumCall, 0, name),
      FunctionAt$_A$(lineNumFunc, name)
\end{lstlisting}

Finally, \inblue{we can make the same query} \rcode{ ?- CallEdge(X, Y)} \inblue{as we did for} the
monolingual programs in Section~\ref{lab:ovmono}. \inblue{Note that this query is also valid
for the multilingual program since we specifically defined the new rule to derive the fact having
the same name,} \rcode{ CallEdge}. The query system now produces
\rcode{ (X, Y) $\in$ \{(6, 2), (3, 7), } \rcode{(8, 4)\}} as the function call
relation results, which include not only \inblue{intra-language} call relations but also
\inblue{inter-language} call relations.

This simple example shows what our extension for multilingual program analysis
requires. Our extension reuses most components in an existing declarative
static analysis with slight modifications. Because the modifications of a
database and language-specific rules are trivial, we can fully automate the
modification process.  The only manual part in our extension is to define
additional language-interoperation rules. We explain how we extend CodeQL to
support multilingual program analysis for Java-C and Python-C programs in the
next section.

\abstract[Summary]{
Declarative static program analysis has become one of the widely-used program
analysis techniques.  Declarative static analyzers perform three steps:
creating databases of facts from program source code, evaluating rules to
generate new facts, and running queries over facts to extract all information
related to specific properties via query systems.  Declarative static analyzers
can easily target diverse programming languages by modifying only databases and
rules for new languages. Because query systems are independent of programming
languages, they are reusable for new languages.  However, even when declarative
analyzers support multiple programming languages they do not support the
analysis of multilingual programs written in two or more programming languages.

We propose a systematic methodology that extends a declarative static analyzer
supporting multiple languages to support multilingual programs as well.
The main idea is to reuse existing components of the analyzer.  Our approach
first generates a merged database of facts, consisting of multiple logical
language spaces. It allows existing language-specific rules to derive new facts
for the corresponding language from the facts in the corresponding language
space.  Then, it defines language-interoperation rules that handle the language
interoperation semantics.  Finally, it uses the same query system to get
analysis results leveraging the language interoperation semantics.  We develop
a proof-of-concept declarative static analyzer for multilingual programs by
extending CodeQL, which can track dataflows across language boundaries. Our
evaluation shows that the analyzer successfully tracks dataflows across Java-C
and Python-C language boundaries and detects genuine interoperation bugs in
real-world multilingual programs.
}

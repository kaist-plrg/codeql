\section{Introduction}

\inred{
Declarative static analysis is a paradigm of static analysis
for detecting bugs and security vulnerabilities in programs.
It specifies
``what'' the program analysis should result in, rather than ``how'' the analysis is performed or the result is calculated.
A declarative static analysis performs three steps.
First, it transforms a program source code to a \textit{database}
of \textit{facts}, the ordered tuples of values. Second, it generates new facts from known facts by applying
{\it rules}. Each rule specifies that when a certain set of facts satisfying a
certain condition is included in the database, then a new certain fact also can be added to the database.
Finally, it extracts information related to specific properties
by applying {\it queries} to the known facts via a query system. The gathered information can then
be interpreted as the final result of the analysis.

Thanks to its declarative characteristic,
declarative static analysis has become one of the widely-used techniques.
for alleviating the challenges and burdens in crafting a static program analyzer.
Datalog~\cite{doop, allen2015D, allen2015stagedD, alpuente2010D,
dawson1996D, naik2006D, reps1994D, smaragdakis2014D, whaley2005D} has
become a promising domain specific language for static analysis~\cite{scholz2016}.
The \doop framework~\cite{doop} using Datalog showed
dramatic speedups for the Java points-to analysis.  The
modularity and declarativeness of Datalog rules are keys in performance
improvement~\cite{doopWorkshop}.  Researchers also have improved static
analysis backends using declarative language engines~\cite{whaley2005D,
hoder2011muz, souffle, madsen2016}.  Avgustinov et al.~\cite{ql2016} proposed
QL, a declarative and object-oriented query language that can be compiled into
Datalog and runs on a relational database. CodeQL~\cite{codeql} is a declarative semantic code analysis engine
based on QL, maintained by Github. It performs dataflow analysis to detect security
vulnerabilities in Java programs, and is scalable to large-size programs with millions of lines of
code. Recently, Glean~\cite{glean}, an experimental
declarative query system, has been developed by Meta, on which users can query
information about code structures for JavaScript and Hack programs.
}

Declarative static analyzers have broadened their analysis targets
\inred{from a single programming language}
to diverse programming languages. To support a new language, one should consider
the three steps of declarative static analysis. Because the query
system is independent of analysis target languages, one can reuse the
existing query system for the new language. Therefore, the first step
is to define a new schema of the database containing facts for the new
language and implement a
front-end component that transforms programs written in the new target language into facts.
Then the next step is to define new language-specific rules for the new target language.  With these
modifications, declarative analyzers can analyze programs written in the new
target language, using the existing queries and the query system.  DOOP currently
supports Python program analysis to detect tensor shape mismatching bugs in
TensorFlow-based Python ML models~\cite{lagouvardos2020static}. \inred{CodeQL now} can track dataflows not only
in Java but also in C++, C\#, JavaScript, Ruby, and Python programs.
Furthermore, Glean plans to support diverse target programming languages,
including Python, Java, C++, Rust, and Haskell.

While the analyzers support multiple programming languages, they do not directly
support the analysis of \inred{\textit{multilingual programs}, the programs} written in two or more
programming languages. Multilingual programs are now widely developed in
various application domains~\cite{kochhar2016large, mergendahlcross}. However,
multilingual programs are often vulnerable to bugs or security issues more than
monolingual programs. A large-scale study on the code quality of multilingual
programs~\cite{kochhar2016large} reported that using multiple languages
together correlates with higher error-proneness. Moreover,
Grichi et al.~\cite{grichi2020impact} showed that two or three times more bugs and security
issues had been reported in the language interoperation than in the
intraoperation in widely-used open-source JNI programs such as OpenJ9 and VLC.

In this paper, we propose a systematic methodology that extends a
declarative static analyzer supporting multiple languages to support
multilingual program analysis as well. Our goal is to maximize the reuse of
already existing components. First, it generates a merged database of facts that can
be separated into multiple logical language spaces.  Each language space
consists of original facts from its corresponding language database, and existing
language-specific rules derive new facts from the facts in the corresponding
language space. Then, to handle the language interoperation semantics in
multilingual programs, we define language-interoperation rules referring to
the language interoperation semantics. The language-interoperation rules derive new
facts from the facts across language spaces. The extensions enable the query system
to extract facts in multilingual programs, taking the language
interoperation semantics into account.
Finally, the same query is evaluated under the same query system to get
analysis results.

To evaluate the practicality of our approach, we develop a proof-of-concept
declarative static analyzer for multilingual programs by extending CodeQL. Our
analyzer tracks dataflows across language boundaries for two types of
multilingual programs: Java-C programs written in Java and C and Python-C
programs written in Python and C. The extension is simple enough in that
it requires only a few lines of automated modifications of CodeQL and additional
language-interoperation rules. We also implement a bug checker on top of
the analyzer to detect dataflow-related interoperation bugs in real-world Java-C programs.
The evaluation shows that our tool successfully tracks dataflows across
Java-C and Python-C language boundaries and detects genuine interoperation
bugs in real-world multilingual programs.

The contributions of this paper are as follows:
\begin{itemize}
\item We propose a systematic methodology that extends a declarative static analyzer
supporting multiple languages to support multilingual program analysis.

\item We implement a proof-of-concept declarative static dataflow analyzer\inred{, \ours,} for two
types of multilingual programs, Java-C and Python-C, by extending CodeQL.

\item We show the practical usefulness of \inred{\ours} in the sense that it detects
dataflow-related bugs at language boundaries of real-world multilingual
programs.
\end{itemize}

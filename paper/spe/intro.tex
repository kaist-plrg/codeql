\section{Introduction}
Advances in modern declarative logical specification languages such as
\inblue{Datalog,} Souffl\'{e}~\cite{souffle} and QL~\cite{ql2016} have facilitated
\emph{declarative static analyses.} A declarative static analysis specifies
rules for ``what'' the analysis results in rather than defining algorithms for
``how'' the analysis is performed.  Typically, it completes three steps.
First, it transforms a program source code into a database containing
\textit{facts} that are tuples of values.  Second, it derives new facts from
known facts in the database by applying {\it rules}.  Each rule specifies a set
of base facts and a derived fact that will be added to the database if
satisfying all the base facts.  Finally, the analysis applies {\it queries} for
specific properties via a query system and produces the results of the queries
as its analysis results.

Thanks to its declarative characteristic, declarative static analysis has
become one of the widely-used techniques.  Datalog \inblue{is} a promising
domain-specific language for static analysis~\cite{doop, codequest, allen2015D,
allen2015stagedD, alpuente2010D, dawson1996D, naik2006D, reps1994D,
smaragdakis2014D, whaley2005D, scholz2016} \inblue{as it can alleviate} challenges and
burdens in crafting a static program analyzer.  For example, \doop~\cite{doop},
a static program analyzer using Datalog, shows remarkable scalability for the
Java points-to analysis, and the scalability comes from the modularity and
declarativeness of Datalog~\cite{doopWorkshop}.
\inblue{As a result of successful declarative static analysis using Datalog,
various declarative static analyzers that are inspired by or based on Datalog
have emerged.} A declarative semantic code
analysis engine maintained by Github, called CodeQL~\cite{codeql}, is scalable
enough to detect security vulnerabilities in Java programs of over millions of
lines of code.  Recently, Meta developed Glean~\cite{glean}, an experimental
declarative query system on which users can query information about code
structures on a large-scale codebase for JavaScript and Hack programs.

Declarative static analyzers have broadened their analysis targets from a
single programming language to diverse programming languages.  To support a new
language, one should consider the three steps of declarative static analysis.
Because the query system is independent of analysis target languages, one can
reuse the existing query system for the new language. Therefore, the first step
is to define a new schema of the database containing facts for the new language
and implement a front-end component that transforms programs written in the new
language into facts.  Then the next step is to define new language-specific
rules for the new language.  With these modifications, declarative analyzers
can analyze programs written in the new language, using the existing queries
and the query system.  \doop~currently supports Python program analysis to
detect tensor shape mismatching bugs in TensorFlow-based Python ML
models~\cite{lagouvardos2020static}.  CodeQL now can track dataflows not only
in Java but also in C++, C\#, JavaScript, Ruby, and Python programs.
Furthermore, Glean plans to support diverse programming languages, including
Python, Java, C++, Rust, and Haskell.

While the analyzers support multiple programming languages, they do not
directly support the analysis of \textit{multilingual programs} written in two
or more programming languages.  Multilingual programs are now widely developed
in various application domains~\cite{kochhar2016large, mergendahlcross}.
However, multilingual programs are often vulnerable to bugs or security issues
more than monolingual programs. A large-scale study on the code quality of
multilingual programs~\cite{kochhar2016large} reported that using multiple
languages together correlates with higher error-proneness. Moreover, Grichi et
al.~\cite{grichi2020impact} showed that two or three times more bugs and
security issues had been reported in the language interoperation than in the
intraoperation in widely-used open-source JNI programs such as OpenJ9 and VLC.

In this paper, we propose a systematic methodology that extends a declarative
static analyzer supporting multiple languages to support multilingual program
analysis as well. Our goal is to maximize the reuse of already existing
components. First, it generates a merged database of facts that can be
separated into multiple logical language spaces.  Each language space consists
of original facts from its corresponding language database, and existing
language-specific rules derive new facts from the facts in the corresponding
language space. Then, to handle the language interoperation semantics in
multilingual programs, we define language-interoperation rules referring to the
language interoperation semantics. The language-interoperation rules derive new
facts from the facts across language spaces. The extensions enable the query
system to extract facts in multilingual programs, taking the language
interoperation semantics into account.  Finally, the same query is evaluated
under the same query system to get analysis results.

To evaluate the practicality of our approach, we develop a proof-of-concept
declarative static analyzer, called \ours, for multilingual programs by
extending CodeQL. \ours tracks dataflows across language boundaries for two
types of multilingual programs: Java-C programs written in Java and C and
Python-C programs written in Python and C. The extension is simple enough in
that it requires only a few lines of automated modifications of CodeQL and
additional language-interoperation rules. We also implement a bug checker on
top of \ours to detect dataflow-related interoperation bugs in real-world
Java-C programs.  The evaluation shows that our tool successfully tracks
dataflows across Java-C and Python-C language boundaries.  The evaluation shows
that \ours is scalable; it takes only about 12 minutes to analyze multilingual
programs of three million lines of code, which the state-of-the-art analyzer
fails to analyze in eight hours.  Also, it detects 33 genuine interoperation
bugs in real-world Java-C programs, including 12 new bugs that the existing
analyzer could not detect due to the lack of scalability.

The contributions of this paper are as follows:
\begin{itemize}
\item We propose a systematic methodology that extends a declarative static
analyzer supporting multiple languages to support multilingual program
analysis.

\item We implement a proof-of-concept declarative static dataflow analyzer,
called \ours, for two types of multilingual programs, Java-C and Python-C, by
extending CodeQL.

\item We show that \ours can successfully detect dataflow-related bugs at
language boundaries of real-world multilingual programs, including new bugs
that the state-of-the-art analyzers could not detect due to the lack of
scalability.

\end{itemize}

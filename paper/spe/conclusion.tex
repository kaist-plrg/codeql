\section{Conclusion}\label{sec:conclude}
Declarative static analysis has become a widely-used analysis technique but
has not supported multilingual programs actively developed in diverse
application domains.
In this paper, we present a practical extension methodology for a declarative static
analyzer that supports multiple languages to analyze multilingual programs.
The first step is to create a merged database consisting of multiple logical language
spaces. Each language space stores facts transformed from source code written in its
corresponding language.
Then, the second step is to define language-interoperation rules to derive facts
across language boundaries.
Our prototype implementation, \ours built on top of CodeQL, successfully tracks dataflows
over language boundaries in both Java-C and Python-C programs.
Using \ours, we found 33 true bugs and vulnerabilities from real-world
JNI applications, 12 of which are from the applications that \lees,
the state-of-the-art Java-C program analyzer, failed to analyze \inred{due to the time
or memory scalability.}
We believe that our approach is applicable to various multilingual programs,
beyond Java-C and Python-C, and to multiple types of analyses.

%In this paper, we suggested the methodology of extending a declarative style
%anlyzer that targets multiple languages into a analyzer that can analyze the
%multilingual programs. First we create the database of fact that can be divided
%into multiple logical language spaces, where each language space consists of
%original facts. Then we define additional language-interoperation rules that
%will generate the facts that reflect interoperation semantics between languages.
%Finally, we perform the original query to obtain the analysis result.
%Based on this approach, we build our prototype implementation \ours on top of
%CodeQL.  Using \ours, we could sucessfully perform dataflow analysis on various
%benchmark suites of JNI programs and extension module programs.  Also, we find
%33 true bugs and vulnerabilities from real-world applications, 12 of which are
%from applications that \lees failed to detect.  We believe that our approach is
%applicable to analysis for other kinds of multilingual programs beyond Java-C
%or Python-C, and to various types of analysi beyond dataflow analysis.

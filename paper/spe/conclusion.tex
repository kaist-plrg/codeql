\section{Conclusion}\label{sec:conclude}
Declarative static analysis has become a widely-used analysis technique but has
not supported multilingual programs actively developed in diverse application
domains.  In this paper, we present a practical extension methodology for a
declarative static analyzer that supports multiple languages to analyze
multilingual programs.  The first step is to create a merged database
consisting of multiple logical language spaces. Each language space stores
facts transformed from source code written in its corresponding language.
Then, the second step is to define language-interoperation rules to derive
facts across language boundaries.  Our prototype implementation, \ours built on
top of CodeQL, successfully tracks dataflows over language boundaries in both
Java-C and Python-C programs.  Using \ours, we found 33 true bugs and
vulnerabilities from real-world JNI applications, 12 of which are from the
applications that \lees, the state-of-the-art Java-C program analyzer, failed
to analyze due to the lack of scalability.  We believe that our approach is
applicable to various multilingual programs, beyond Java-C and Python-C, and to
multiple types of analyses.

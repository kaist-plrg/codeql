\documentclass{letter}
\usepackage{graphicx,fullpage}
\pagestyle{empty}

\signature{
Dongjun Youn\\
Ph.D Candidate\\
School of Computing\\
KAIST
%Korea Advanced Institute of Science and Technology (KAIST)
}
\address{
School of Computing\\
Korea Advanced Institute of Science and Technology (KAIST)\\
291 Daehak-ro, Yuseong-gu\\
Daejeon, Republic of Korea 34141
}
\date{August 19, 2022}
\begin{document}
\begin{letter}{
Software: Practice and Experience
}

\vspace*{-10em}
\hspace*{-3em}
\noindent
\makebox[0pt][l]{%
  \raisebox{-\totalheight}[0pt][0pt]{%
    \includegraphics[width=1.1\textwidth]{socheader}}}
\vspace*{10em}

\opening{Dear Editors:}

We are submitting a reserach article manuscript entitled: ``Declarative Static Program Analysis
for Multilingual Programs'' for exclusive consideration of publication as an
article in Software: Practice and Experience.
This article presents a systematic methodology that extends a declarative static
analyzer to support multilingual programs.  The first step is to create a
merged database consisting of multiple logical language spaces. Each language
space stores facts transformed from source code written in its corresponding
language.  Then, the second step is to define language-interoperation rules to
derive facts across language boundaries.  We develop a proof-of-concept
declarative static analyzer for multilingual programs by extending CodeQL,
which can track dataflows across language boundaries. Our evaluation shows that
the analyzer successfully tracks dataflows across Java-C and Python-C language
boundaries and detects genuine interoperation bugs in real-world multilingual
programs.  We believe that our approach is applicable to various multilingual
programs, beyond Java-C and Python-C, and to multiple types of analyses.

\medskip

Thank you for your consideration of our work.
Please address all correspondence concerning this manuscript to Dongjun Youn at KAIST and
feel free to correspond with him by e-mail ({\tt f52985@kaist.ac.kr}).

\closing{Sincerely,}
\end{letter}
\end{document}

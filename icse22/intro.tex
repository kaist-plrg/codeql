\section{Introduction}
Multilingual programs have been widely developed in various application
domains. Multilingual programs are written in multiple different programming
languages and the languages interoperate each other via a foriegn function
interface (FFI).  Taking benefits of each programming language, developers
often implement complex programs productively, improve performance of their
programs, and reuse existing modules written in different programming
languages. Kochhar et al. reported that about half of most popular open-source
projects mainly written in one of popular programming languages are
multilingual~\cite{kochhar2016large}.


%There are many cases where a single language alone is not enough to meet the
%requirement of large and comlplex programs nowadays, and using multiple
%languages is one of the promising way to implement such programs. For this
%reason, many languages support FFI(Foreign function interface), which enables a
%program written in a language to call the function written in another language.
%For example, (Write examples of FFI here)

However, multilingual programs are often vulnerable to bugs or security issues
more than monolingual programs.  Language interoperation on the FFI is complex
itself and has different semantics from the languages' semantics.  Programmer
errors on the language interoperation frequently occur but are hard to get
detected and fixed, because development tools such as IDEs and compilers cannot
detect erroneous behaviors on the language interoperation.  Previous research
conducted a large-scale study of multilingual programs' code quality, and
showed that using multiple languages together correlates with higher
error-proneness~\cite{kochhar2016large}.  Moreover, Grichi et.al.  reported
that two or three times more bugs and security issues have been reported on the
language interoperation than intraoperation in widely used open-source JNI
programs, such as OpenJ9 and VLC, written in both Java and
C~\cite{grichi2020impact}.

%However, writing a relialbe multilingual program is not an easy task. The
%programmer of multilingual programs should fully understand semantics of both
%language, in addition to the inter-operation semantics between two languages in
%order to write the correct code.  Such complex semantics make it much more
%difficult to reson about the program, and programmers may leave some bugs that
%can crash the program in runtime, or security vulnerabilities that attackers
%can take advantage of.

Static analysis is one of promising solutions to detect bugs and security
vulnerabilities in monolingual programs, but analyzing multiple different
language semantics together is challenging in multilingual programs.
Researchers have proposed to leverage existing general purpose static analyzers
for monolingual programs~\cite{JN-SAF, LeeASE20}. They first analyze each part
written in a single language using its corresponding analyzer to generate its
summary as an abstraction of behaviors of the part. Then, they perform a target
analysis to detect bugs and secrutiy vulnerabilities cross language boundaries
using the generated summaries. While the approaches can analyze different
semantics of different languages and detect bugs and security vulnerabilities
in multilingual programs, they often suffer from performance overhead. Because
they leverage the full features of expensive general purpose static analyzers,
they may waste their analysis time on the features that may not be relevant to
summary generation for the target analysis.


%Performing static analysis on multilingual programs can help developers to
%write more reliable code. (Write importance of static analysis here.)
%
%For this reason, previous researchers have tried to implement the analyzer for
%multilingual analysis with various approaches.  One approach[1][2] is
%translation-based approach, where one language is translated or compiled into
%another language and analysis on translated language is performed with existing
%analyzer.  This approach has a virtue in that existing analyzers for tranalated
%language can be reused with almost no modification.  However, since translating
%one language into another one while preserving every semantics is a challenging
%task, the translation is usually performed in abstract manner to approximate
%the effect of original code might have, and translated code is composed of the
%subset of the target langague. Another approach[3] is combination-based
%approach, which is to modify and combine existing analyzers of each language,
%so that two analyzers give the analysis result in unified output format, and
%the analysis is performed with respect to this output format.

%A common, inevitable pattern of implementing analyzer for multilingual program
%is that, due to the nature of multilingual programs being composed of multiple
%language, the analyzer itself also consists of multiple components. The problem
%of the previous approaches is that each of the component treats other
%components as blackbox. The detailed implementation of a component is not
%visible to another component, and the only way one component can interact with
%each other is through a fixed form of API. For example, in translation-based
%approach, the translator and the client-analyzer can only communicate through a
%syntax of target language, and in combination-based approach, two analyzers
%communicate via unified output format. This makes the implementation
%restrictive and inflexible. Modifying one component requires that it still
%meets the API requirement after the modification. Otherwise, modification to
%multiple components should be made simultaneously, which can be imposes extra
%burden on analyzer developers, especially if each component is very different
%from each other. For example, we can consider a situation where one wants to
%extend the client-analysis so that it performs more accurate analysis. If the
%current API does not contain enough information, then it might be impossible to
%implement such a extended anlysis that the developer is thinking of. Then it is
%required that API is extended and the other components pass such information
%via API, which requries more work for the developer.

%In this paper, we suggest a novel methodology for implementing static analysis
%of multilingual program, which is using declarative language.  In declarative
%language, everything is expressed as in terms of "data-fact" and "rule".
%"Data-fact" is simply an ordered tuples of elements, and "rule" is a way to
%define how one data-fact can be derived from other data-facts. Evaluating
%declarative program is defined as finding all possible data-facts that can be
%derived by given basic data-facts and rules.  Adopting this approach, we can
%perform static analysis on multilingual programs in following sense.  The first
%step is to converted each language into a set of syntactic data-facts.
%Programmers can define the intra-language rules that would generate new
%data-facts based on already known data-facts. New data-facts generated in this
%way would tend to contain more higer-level information, considering the
%semantics of each languge. Finally, programmer can implement the static
%analysis by defining the inter-language rules considering the inter-operation
%semantcis between two languages, and evaluating these new rules along with
%intra-language rules would give the final analysis result in form of a set of
%data-facts.

In this paper, we propose a novel approach that statically analyzes
multilingual programs using a declarative-style analysis for the first time.
Declarative-style analysis consists of \textit{data-facts} and \textit{rules}.
A data-fact is simply an ordered tuples of elements, which indicates the
information that corresponding elements are in certain relation. A Rule defines
how one data-fact can be derived from other data-facts. Evaluating declarative
program is performed by finding possible data-facts that can be derived by
given data-facts and rules.  Adopting this approach, we can perform static
analysis on multilingual programs in following sense.  The first step is to
convert each language into a set of syntactic data-facts.  Then, Programmers
can define the intra-language rules that would generate new data-facts based on
already known data-facts. Finally, programmer can implement the static analysis
by defining the inter-language rules considering the inter-operation semantcis
between two languages. Evaluating these new rules along with intra-language
rules would give the final analysis result in form of a set of data-facts.
The biggest advantage of using declarative-stlye analysis is that it uses
query-based analysis, which means that the target-analysis is written in
special rules called queries, and only datafacts that is required to evaluate
the query is calculated. Unlike the previous approaches where general purpose
analyzers are used to generate the summaires and unnecessary computation might
be performed during this process, our approach does not need to spend time
on generating unnecessary sumamry or calculating the unrelated data-facts.

%In declarative
%language, everything is expressed as in terms of "data-fact" and "rule".
%"Data-fact" is simply an ordered tuples of elements, and "rule" is a way to
%define how one data-fact can be derived from other data-facts. Evaluating
%declarative program is defined as finding all possible data-facts that can be
%derived by given basic data-facts and rules.  Adopting this approach, we can
%perform static analysis on multilingual programs in following sense.  The first
%step is to converted each language into a set of syntactic data-facts.
%Programmers can define the intra-language rules that would generate new
%data-facts based on already known data-facts. New data-facts generated in this
%way would tend to contain more higer-level information, considering the
%semantics of each languge. Finally, programmer can implement the static
%analysis by defining the inter-language rules considering the inter-operation
%semantcis between two languages, and evaluating these new rules along with
%intra-language rules would give the final analysis result in form of a set of
%data-facts.


%On top of the know advantages of enabling fast, modular and scalable
%implementation[4], using declarative language can also be helpful to alleviate
%the problems of multilingual analysis mentioned above. The main reason is that
%in declarative style, any rule can easily utilze the data-facts generated by
%any component, that is, no fixed API is needed between two components.  This
%feature lessens the burden of modifying or extending the existing analyzer.
%The additionaly required task is boiled down to defining some more rules and
%data-facts for different components, which does not impose significant overhead
%than implementing the analyzer itself.

To show the usefulness of our approach, we implement a proof-of-concept static
analyzer for JNI programs written in both Java and C. JNI programs have been
actively developed for both desktop and mobile applications, but they often
contain diverse bad smells in code, which may introduce bugs and security
issues~\cite{nishiwaki2012sean, grichi2019state, abidi2019behind,
abidi2021multi}. We implement our analyzer on top of CodeQL~\cite{codeql}, a
static analysis engine powered by a declarative language named QL.  It
abstracts away the need of managing underlying data structure or implementing
the efficient algorithm for evaluating the query, thus enabling simple and fast
query writing. We tested our analyzer with real-world android JNI applications
to see whether the performance of our analyzer is comparable with existing works.
Surprisingly, our experiment result show that our analyzer outperforms the
state of the art JNI analyzer in terms of scalability in that our analyzer
shows x14 speed up on average, while shoinwg no analysis precision degrade. In
addition,  we could easily emulate the various kinds of bug detectors
illustrated in previous researches on top of our analyzer, thanks to the
power of the CodeQL. Using these bug checkers, we could find 25 new bugs and
vulnerabilites from the real world applications.

%the practicality of our approach, we chose JNI programs as analysis
%target, and implemented dataflow analysis as client-analysis using CodeQL[5], a
%static analysis where implementation can be written in declarative style. We
%applied our analyzer to real-world android JNI applications retrieved from
%F-droid[6], a repository for open-source android applications to see whether
%performance of this implementation is competable with existing works.
%Surprisingly, our experiment result show that our implementation outperforms
%the state of the art JNI analyzer in terms of speed and analyzer our
%implementation shows x14 speed up on average, and shoinwg the slightly better
%precision. In addition, we could emulate the various kinds of bug detectors
%mentioned in previous researches on top of our analyzer in declarative style
%with relative ease, and could find the genuine bugs and vulnerabilites from the
%benchmark.

%The contribution of this paper is as follows:
%
%+ We suggested a new methodology of using declarative style static analysis
%for multilingual program, which is more flexible and extensible.
%
%+ We implemented JNI dataflow analyzer with CodeQL based on declarative-style approach,
%which outperforms state-of-art analyzers in terms of scalability and precision.
%
%+ We could find new bugs and vulnerabilities from the real-world JNI applications
%by easily implementing the bug detectors from the previous researches.

The contributions of this paper include the following:

\begin{itemize}
  \item We propose a simple and effective declarative-style static analysis for
    multilingual programs for the first time. Our approach analyzes dataflows
    over language boundaries seamlessly without complex and expensive summary
    generation.

\item We implement a declarative-style JNI program analyzer on top of CodeQL.
  The design of our analyzer enables us to easily implement various bug
    checkers for JNI programs using the powerful query language of CodeQL. 

\item We show practical usefulness of the JNI program analyzer for real-world
  JNI programs. Our tool detected the same types of bugs and security
    vulnerabilities with higher precision 14x faster on average than the
    state-of-the-art JNI program analyzers.
\end{itemize}

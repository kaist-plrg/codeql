\section{Introduction}
There are many cases where a single language alone is not enough to meet the requirement of large and comlplex programs nowadays,
so using multiple languages is one of the promising way to implement such programs. For example ~~~

However, writing a relialbe multilingual program is not an easy task. (Too similar with ASE?)

Performing dataflow analysis of multilingual programs help developers to write more reliable code.

Many researchers have tried to perform static (dataflow?) analysis for multilingual programs.
(JN-SAF, Illia, ASE)

In this paper, we suggest a novel methodology for performing dataflow analysis of multilingual program.
(Which uses lightweight declarative style)

To evaluate the feasibility of our approach, we implemented the dataflow analysis for JNI programs using CodeQL.
(Which showed better performance)

We also tested out analyzer with real world applications retrieved from F-droid, and found new bugs and vulnerabilities from these apps.

The contribution of this paper is as follows:

+ We suggested a new methodology of using declarative style dataflow analysis for multilingual program

+ We implemented JNI dataflow analyzer with CodeQL, which outperforms state-of-art analyzers in terms of speed and precision.

+ We found and reported new bugs and vulnerabilities from the real-world JNI applications.

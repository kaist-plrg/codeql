\section{Related Work}
%We propose the declarative-style static analysis of multilingual programs and
%actualize our approach as the JNI dataflow analyzer.
We describe three closely related categories of work below.
\subsection{Analysis for JNI}
Among various language representations of JNI programs, studies dealing with
pairs of languages are most closed related.  For JNI programs consisting of
Java and C source codes, ILEA~\cite{ILEA} extends the Java Virtual Machine
Language (JVML) and Jlint\footnote{\url{http://artho.com/jlint/manual.html}},
which is a JVML dataflow analyzer, to integrate C codes into one language.
Since a modest extension of JVML cannot fully support C semantics like memory
models, it extremely over-approximates C operations such as read and write
through pointers.  Lee et al.~\cite{LeeASE20} proposed a general approach to
analyze multilingual programs in the form of host and guest languages.  Their
approach first analyzes a guest language (C, in their implementation) to extract
semantic summaries (by Infer\footnote{\url{https://fbinfer.com/}}), translates
and integrates the summaries into a host language (Java), and then utilizes host
language analyzer (Flowdroid~\cite{Flowdroid}).  While their approach can
leverage the full features of existing static analyzers, ours inherits the
query-based analysis and can compute target properties effectively without
expensive and redundant summary generation.

Other studies deal with binaries rather than C/C++.
Fourtounis et al.~\cite{scanning} proposed a lightweight reverse engineering
technique to recover Java method calls in binary codes instead of performing
heavy analyses on binary code.  Their reverse engineering generates the data
facts of the Java calls with declarative style analyses for Java.
Their reverse engineering is lightweight yet targeting on Java method calls in
binary, in contrast, our approach targets on two general declarative style
analyses for each language and their interoperations seamlessly.
JN-SAF~\cite{JN-SAF} defines a unified summary in order to handle Java bytecode
and binary at once.  It extracts the summaries for each method with a Java
bytecode dataflow analyzer and a binary symbolic execution and composes the
summaries in a bottom-up manner to detect security vulnerabilities.
It sacrifices propagating dataflow facts until a fixed point by breaking a cycle
in the call graph arbitrarily to mitigate expensive computations from the
dataflow analysis and symbolic execution.  Our approach is scalable for the JNI
dataflow analysis even without sacrificing fixed point computations when C
source code is available.

\inred{Security analysis, dynamic analysis}

\subsection{Analysis for the other FFIs}
Android applications have another FFI between Java and JavaScript to maximize
their portability.  Android hybrid apps use JavaScript code to be run on
multiple browsers and can communicate with native Java code.
HybriDroid~\cite{HybriDroid} analyzes Java and JavaScript at the same time on
the top of the WALA framework~\cite{WALA} to detect programmer errors on interlanguage
communications and to track private data leakages.  Bae et al.~\cite{BaeICSE19}
tackled the expensive analysis of HybriDroid and proposed a light-weight type
system covering types of bug in HybriDroid.  However, their type system does not
cover tracking flows of private data.  Jin et al.~\cite{jin2014code} proposed a
static detection of code injection attacks from JavaScript to Java.  To avoid
expensive whole multilingual program analyses, they manually model APIs in the
PhoneGap\footnote{\url{https://cordova.apache.org/}} framework and apply a taint
analysis on JavaScript only.

\inred{Security analysis, dynamic analysis}

Python supports the interoperability mechanism with C~\cite{PythonC}.
Developers often import performance-oriented C code to high-level Python code.
Very recent work~\cite{sas2021} proposed a Python-C analyzer by reusing each
analyzer in the Mopsa framework~\cite{Mopsa}.  Their target properties require
precise context-sensitive value analyses.  Our approach easily activates only
the required features of each analyzer for such target properties.

\inred{(type systems)[12, 13, 14] OCaml, Java, C}

\subsection{Declarative-style analysis}
Declarative-style analysis alleviates challenges in crafting a static program
analyzer and thus Datalog \inred{[ten papers]} has become a dominant
DSL for static analysis~\cite{scholz2016}.
Bravenboer and Smaragdakis proposed the \doop framework~\cite{doop} using Datalog
that showed dramatic speedups for the Java points-to analysis.  They explain that
the modularity and declarativeness of Datalog rules are the keys to improve
performance~\cite{doopWorkshop}.  In addition, researchers have improved
analysis backends, declarative language engines%~\cite{bddbddb[37], µZ[16], Soufflé, madsen2016}.

Avgustinov et al.~\cite{ql2016} proposed QL, a declarative and object-oriented
query language to be compiled into Datalog and runs on a relational DB, and static
analyzers implemented QL which scale to millions of lines of code.
Our approach successfully inherits the fast and easy analysis of declarative-style
and emphasizes the benefits with the problem of multilingual program analysis.

\begin{figure}[t]
  \centering
  \vspace{2mm}
  \includegraphics[width=0.47\textwidth]{img/codeql.pdf}
  \caption{Overall structure of JN-QL}
  \label{fig:codeql}
\end{figure}

\section{Implementation}\label{sec:impl}
In this section, we present JN-QL, a prototype implementation of our
approach to a declarative static analysis for multilingual programs,
in the case of dataflow analysis for JNI programs using CodeQL~\cite{codeql}.

\subsection{CodeQL}
CodeQL is a static analysis engine that transforms source code
programs into databases, and performs analysis by evaluating queries
written in the declarative language called QL (Query Language).
In QL, one can define rules by defining ``predicates'' or ``classes.''
For example, the following QL code defines the predicate \codeql{isOneTwoThree}:
\begin{lstlisting}[style=codeql,xleftmargin=2.5em]
predicate isOneTwoThree(int n) {
  n = 1
  or
  n = 2
  or
  n = 3
}
\end{lstlisting}
and the following definition defines the class \codeql{OneTwoThree}:
\begin{lstlisting}[style=codeql,xleftmargin=2.5em]
class OneTwoThree extends int {
  OneTwoThree() { // characteristic predicate
    this = 1
    or
    this = 2
    or
    this = 3
  }
}
\end{lstlisting}
A class defines a set of elements that satisfy the predicate called
``\codeql{characteristic predicate}.''
For more information about QL, one can refer to \citet{ql2016}'s paper
or the official document~\cite{codeql}.

Figure~\ref{fig:codeql} presents the overall structure of JN-QL.
First, it generates databases for both languages, C\footnote{
Even though JN-QL analyzes JNI programs written in Java and both C and
C++, this paper refers to C only fore presentation brevity.} and
Java, and merges them to one database.  This corresponds
to the step of extracting syntactic datafacts from source code programs.
Then, we merge the common datafacts and rules in both languages,
which are parts of their libraries that CodeQL provides, into one merged library.
Finally, using the merged database and merged library, a user can write a query to
perform a client-analysis, and evaluate it to produce its analysis result.

\subsection{Creating Databases}
For compiled languages such as C and Java, CodeQL generates their database
by compiling source programs.  When a compiler compiles a program,
CodeQL monitors the compiler to extract necessary information and
creates a database with the extracted information.
CodeQL creates a database for a single language in two steps:
1) it stores the extracted information from a compiler in a trap
file, a human readable file format, and 2) it finalizes trap files
into databases in the binary format. For example, the
following shows a sample trap file:

\begin{lstlisting}[style=java,numbers=none]
#10017=@"class;hello.HelloJNI"  ...
#10044=@"type;int"
primitives(#10044,"int")  ...
#10061=@"callable;{#10017}.java_callback({#10044}){#10044}"  ...
#10067=@"params;{#10061};0"
params(#10067,#10044,0,#10061,#10067)
paramName(#10067,"java_callback_param")
\end{lstlisting}
which describes the parameter information about \javacode{java\_callback}.

\begin{figure}[t]
  \centering
  \vspace{2mm}
  \begin{subfigure}[t]{0.5\textwidth}
\begin{lstlisting}[style=codeql,xleftmargin=2.5em]
class Node extends TNode { ... }
predicate simpleLocalFlowStep(Node from, Node to) {
  exprToExprStep_nocfg( // Expr -> Expr
    from.asExpr(),
    to.asExpr()
  )
  or ...  // Assignment -> LValue post-update node
}
\end{lstlisting}
    \vspace*{-.5em}
    \caption{c/dataflow/internal/DataFlowUtil.qll}
  \end{subfigure}
  \begin{subfigure}[t]{0.5\textwidth}
\begin{lstlisting}[style=codeql,xleftmargin=2.5em]
class Node extends TNode { ... }
predicate simpleLocalFlowStep(Node node1, Node node2) {
  // Variable flow steps through
  // adjacent def-use and use-use pairs.
  exists(SsaExplicitUpdate upd |
    upd.getDefiningExpr().(VariableAssign).getSource()
      = node1.asExpr() or
    upd.getDefiningExpr().(AssignOp) = node1.asExpr()
  |
    node2.asExpr() = upd.getAFirstUse()
  )
  or ...  // Flow through this
}
\end{lstlisting}
    \vspace*{-.5em}
    \caption{java/dataflow/internal/DataFlowUtil.qll}
  \end{subfigure}
  \vspace*{-.5em}
  \caption{QL libraries for C and Java}
  \label{fig:qll}
\end{figure}

To create a single database for both C and Java, JN-QL maintains
a separate trap file for each language and then merges them.
Note that both trap files may have tables with the same name.
To avoid name conflicts in a merged database, we add a
language-specific prefix to each table.
For example, if both trap files have tables named \codeql{@expr},
we rename the table from C as \codeql{@c\_expr},
and the table from Java as \codeql{@java\_expr"}.
After renaming such tables, we can safely finalize the trap files into
databases on which queries can be evaluated.

\subsection{Lifting Libraries}

CodeQL provides various libraries for C and Java, including
pre-defined predicates and classes for users to implement their own analyses.
A dataflow analysis library is such a library, which supports both C and Java.
For example, Figure~\ref{fig:qll} shows sample QL libraries for C and Java,
which uses the same class name \ccode{Node} and the same predicate
name \ccode{simpleLocalFlowStep}.
However, even with the same name, the class \ccode{Node} in C and the
class \javacode{Node} in Java are different classes, which are incompatible.
The same applies to the predicate \ccode{simpleLocalFlowStep}.
In other words, we can not use the class \ccode{Node} in C as an
argument of the predicate \javacode{simpleLocalFlowStep} in Java or vice versa.

To make classes and predicates in C and Java become compatible,
we lift each library to the common level.
First, we encapsulate each of the original dataflows in a CodeQL
module named C and JAVA so that we can distinguish the original
classes and predicates by lifted ones.
We can lift a class by first defining a sum type, denoting that
the lifted class comes either from C or Java, and then making the
lifted class be of that type.  We also implement two member predicates
that can cast the lifted class into the C or Java class:
\begin{lstlisting}[style=codeql,xleftmargin=2.5em]
private newtype TNode =
  TJavaNode(JAVA::Node n)
  or
  TCNode(C::Node n)
class Node extends TNode {
  JAVA::Node asJava() {
    this = TJavaNode(result)
  }
  C::Node asC() {
    this = TCNode(result)
  }
  ...
}
\end{lstlisting}
Similarly, we can lift a predicate by combining two original predicates with
the \codeql{or} connective. For each original predicate, we cast down
each of the arguments and return values to the class of its corresponding language:
\begin{lstlisting}[style=codeql,xleftmargin=2.5em]
predicate simpleLocalFlowStep(Node node1, Node node2) {
  JAVA::simpleLocalFlowStep(
    node1.asJava(), node2.asJava()
  )
  or
  C::simpleLocalFlowStep(
    node1.asC(), node2.asC()
  )
}
\end{lstlisting}
After lifting, lifted predicates show the equivalent behaviors as the
original ones if all the arguments are from the same language.



\subsection{Merging Libraries}\label{sec:merging}
After lifting libraries for different languages, we extend predicates to
reflect the interoperation semantics between multiple languages.
We identified various interactions from Java to C and from C to Java
and extended predicates to model their behaviors.

\begin{figure}[t]
\begin{lstlisting}[style=codeql,xleftmargin=2.5em]
DataFlowCallable viableCallable(DataFlowCall c) {
  result.asJava() = JAVA::viableCallable(c.asJava())
  or
  result.asC() = C::viableCallable(c.asC())
  or
  result.asC() = viableCallableJ2C(c.asJava())
  or
  result.asJava() = viableCallableC2J(c.asC())
}
\end{lstlisting}
\caption{library merge}\label{fig:merge}
\end{figure}

For example, Figure~\ref{fig:merge} shows how we extend the predicate
named \codeql{viableCallable}.
Lines 2 and 4 show the results of lifting, and they take advantage of the
original predicates from the dataflow libraries.  They handle
intra-language call edges from Java to Java and from C to C.
Lines 6 and 8 show the results of merging libraries, and they are
for inter-language call edges.  The predicate \codeql{viableCallableJ2C} finds call edges
from Java to C, and the predicate \codeql{viableCallableC2J} finds call edges from
C to Java. Because these two call edges have different characteristics, we
implemented them differently.

%\medskip
\textbf{Java to C.} In JNI programs, one can make interactions from
Java to C by calling native functions in C from Java code. The target
of such a function call is determined in a static manner.
The target function should follow the JNI naming convention, which is adding
\codeql{Java\_} as prefix, 
followed by a fully qualified name of its class and the additional \codeql{\_}, to the method name.
For example, the target function name for a function call of
\codeql{cfunction} would be \codeql{Java\_fully\_qualified\_class\_name\_cfunction}.
With this convention, we can define \codeql{viableCallableJ2C}
so that \codeql{f = viablaCallableJ2C(m)} holds when
\codeql{f.toString() = "Java\_" + m.className() + "\_" + "m.toString()} holds.


\medskip
\textbf{C to Java.} The interaction from C to Java is more complex and
requires more careful implementation than the interaction from Java to C.
The primary difference is that a method call from C to Java requires
the runtime values of variables, which may not be always possible.
First, C code calls the interface function \ccode{GetMethodID(name, sig)}
to get the ``method ID'' of the method whose name matches the first argument
and the type signature matches the second argument passed to this function.
This method ID is stored at a variable, say \ccode{mid}, and
an actual method call is invoked by another
interface function, \ccode{Call<type>Method(obj, mid, args...)}. Calling this interface 
function corresponds to calling the method that \ccode{mid} indicates
with \ccode{obj} as ``this object'' and \ccode{args} as the arguments.

To correctly handle this method call, we should be able to answer these
questions: ``When we call \ccode{GetMethodID},
what are the string values of \ccode{name} and \ccode{sig}?'' and
``When we call \ccode{Call<type>Method}, what is the method ID value of \ccode{mid}?''
Answering these questions requires dataflow analysis;
to soundly answer to them, inter-language dataflow analysis is necessary.
However, in practice, intra-language dataflow analysis is enough in most cases.
Therefore, we implemented two ``intra-flow'' analysis modules for C,
which find 1) dataflows from string literals to the arguments of interface functions,
and 2) dataflows from interface function call results to the arguments of interface functions.
Using these modules, we can implement the
predicate \codeql{viableCallableC2J} by adding a call edge from a \ccode{Call<type>Method}
call to the method \ccode{m}, if there is a flow to \ccode{mid} from a
call to \ccode{GetMethodID}, and string values that flow into the
arguments \ccode{name} and \ccode{sig} of \ccode{GetMethodID} that
correspond to the name and the type signature of the method \ccode{m}.

In addition to the predicate \codeql{viableCallable},
we extended more predicates to consider other JNI interface functions such as
\ccode{findClass} and \ccode{GetFieldID}.
Most of such extended predicates are specialized \codeql{step} predicates.
We extended them in a similar way to the calls from C to Java described above.

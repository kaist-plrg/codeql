%%
%% This is file `sample-sigconf.tex',
%% generated with the docstrip utility,
%% and then modified to 'main.tex' by Dongjun.
%%
%% The original source files were:
%%
%% samples.dtx  (with options: `sigconf')
%% 
%% IMPORTANT NOTICE:
%% 
%% For the copyright see the source file.
%% 
%% Any modified versions of this file must be renamed
%% with new filenames distinct from sample-sigconf.tex.
%% 
%% For distribution of the original source see the terms
%% for copying and modification in the file samples.dtx.
%% 
%% This generated file may be distributed as long as the
%% original source files, as listed above, are part of the
%% same distribution. (The sources need not necessarily be
%% in the same archive or directory.)
%%
%%
%% Commands for TeXCount
%TC:macro \cite [option:text,text]
%TC:macro \citep [option:text,text]
%TC:macro \citet [option:text,text]
%TC:envir table 0 1
%TC:envir table* 0 1
%TC:envir tabular [ignore] word
%TC:envir displaymath 0 word
%TC:envir math 0 word
%TC:envir comment 0 0
%%
%%
%% The first command in your LaTeX source must be the \documentclass command.
\documentclass[sigconf,review,anonymous]{acmart}
\acmConference[ICSE 2022]{The 44th International Conference on Software Engineering}{May 21–29, 2022}{Pittsburgh, PA, USA}

%%
%% \BibTeX command to typeset BibTeX logo in the docs
\AtBeginDocument{%
  \providecommand\BibTeX{{%
    \normalfont B\kern-0.5em{\scshape i\kern-0.25em b}\kern-0.8em\TeX}}}

%% Rights management information.  This information is sent to you
%% when you complete the rights form.  These commands have SAMPLE
%% values in them; it is your responsibility as an author to replace
%% the commands and values with those provided to you when you
%% complete the rights form.
\setcopyright{acmcopyright}
\copyrightyear{2018}
\acmYear{2018}
\acmDOI{10.1145/1122445.1122456}

%% These commands are for a PROCEEDINGS abstract or paper.
\acmConference[ICSE 2022]{The 44th International Conference on
   Software Engineering}{May 21–29, 2022}{Pittsburgh, PA, USA}
\acmBooktitle{Woodstock '18: ACM Symposium on Neural Gaze Detection,
  June 03--05, 2018, Woodstock, NY}
\acmPrice{15.00}
\acmISBN{978-1-4503-XXXX-X/18/06}


%%
%% Submission ID.
%% Use this when submitting an article to a sponsored event. You'll
%% receive a unique submission ID from the organizers
%% of the event, and this ID should be used as the parameter to this command.
%%\acmSubmissionID{123-A56-BU3}

%%
%% The majority of ACM publications use numbered citations and
%% references.  The command \citestyle{authoryear} switches to the
%% "author year" style.
%%
%% If you are preparing content for an event
%% sponsored by ACM SIGGRAPH, you must use the "author year" style of
%% citations and references.
%% Uncommenting
%% the next command will enable that style.
%%\citestyle{acmauthoryear}

%%
%% end of the preamble, start of the body of the document source.
\begin{document}

%%
%% The "title" command has an optional parameter,
%% allowing the author to define a "short title" to be used in page headers.
\title{Declarative-Style Dataflow Analysis of JNI Programs using CodeQL}

%%
%% The "author" command and its associated commands are used to define
%% the authors and their affiliations.
%% Of note is the shared affiliation of the first two authors, and the
%% "authornote" and "authornotemark" commands
%% used to denote shared contribution to the research.
\author{Dongjun Youn}
\authornote{Both authors contributed equally to this research.}
\email{trovato@corporation.com}
\orcid{1234-5678-9012}
\author{Sungho Lee}
\authornotemark[1]
\email{webmaster@marysville-ohio.com}
\affiliation{%
  \institution{Institute for Clarity in Documentation}
  \streetaddress{P.O. Box 1212}
  \city{Dublin}
  \state{Ohio}
  \country{USA}
  \postcode{43017-6221}
}

\author{Joonyoung Park}
\affiliation{%
  \institution{The Th{\o}rv{\"a}ld Group}
  \streetaddress{1 Th{\o}rv{\"a}ld Circle}
  \city{Hekla}
  \country{Iceland}}
\email{larst@affiliation.org}

\author{Sukyoung Ryu}
\affiliation{%
  \institution{Inria Paris-Rocquencourt}
  \city{Rocquencourt}
  \country{France}
}

%%
%% By default, the full list of authors will be used in the page
%% headers. Often, this list is too long, and will overlap
%% other information printed in the page headers. This command allows
%% the author to define a more concise list
%% of authors' names for this purpose.
\renewcommand{\shortauthors}{Trovato and Tobin, et al.}

\begin{abstract}
Multilingual programs are written in multiple programming languages
taking advantage of their own language features and benefits.
While multilingual programs have been widely developed for various purposes,
they are more vulnerable to bugs and security issues than monolingual programs.
Thus, researchers have proposed two approaches to analyze multilingual programs.
The first approach leverages existing static analyzers for
monolingual programs to analyze multilingual programs.
It first analyzes each part written in a single language
using its corresponding analyzer to generate its summary and then
perform \textit{a target analysis} to detect bugs and security vulnerabilities cross
language boundaries using the generated summaries.
The second approach abstracts the behaviors of
each part written in a single language into a single form
and performs a target analysis using existing top-down analyzers.
While they can utilize the full features of existing static analyzers,
they often fail to analyze large-scale real-world applications.
Because existing static analyzers support abstractions of all the language features,
they may waste their analysis time on the features that may not be
relevant to summary generation for the target analysis. 

In this paper, we propose a simple and effective declarative-style static
analysis for multilingual programs for the first time.
A declarative-style analysis expresses analysis information as {\it datafacts},
describes how to derive a datafact from other datafacts as {\it rules},
and performs a {\it query-based} analysis considering only relevant datafacts.
Our approach extracts syntactic datafacts from the parts written in different
languages separately and defines additional rules to describe the interaction
between different languages.  Because our approach works directly on datafacts
extracted from programs and inherits the query-based-analysis
feature from a declarative-style analysis, it can effectively check target
properties in multilingual programs without expensive and redundant summary generation.
As a proof-of-concept, we implement a JNI program static analyzer \ours
on top of CodeQL, which detects bugs and security vulnerabilities by tracking
dataflows over language boundaries between Java and C.
Our evaluation shows that \ours is {\it precise} in that it analyzes
dataflows with higher precision and is {\it scalable} in that it analyzes
more applications than the state-of-the-art analyzers.
It also reported 33 genuine bugs in real-world JNI Android applications.
\end{abstract}


%%
%% The code below is generated by the tool at http://dl.acm.org/ccs.cfm.
%% Please copy and paste the code instead of the example below.
%%
\begin{CCSXML}
<ccs2012>
 <concept>
  <concept_id>10010520.10010553.10010562</concept_id>
  <concept_desc>Computer systems organization~Embedded systems</concept_desc>
  <concept_significance>500</concept_significance>
 </concept>
 <concept>
  <concept_id>10010520.10010575.10010755</concept_id>
  <concept_desc>Computer systems organization~Redundancy</concept_desc>
  <concept_significance>300</concept_significance>
 </concept>
 <concept>
  <concept_id>10010520.10010553.10010554</concept_id>
  <concept_desc>Computer systems organization~Robotics</concept_desc>
  <concept_significance>100</concept_significance>
 </concept>
 <concept>
  <concept_id>10003033.10003083.10003095</concept_id>
  <concept_desc>Networks~Network reliability</concept_desc>
  <concept_significance>100</concept_significance>
 </concept>
</ccs2012>
\end{CCSXML}

\ccsdesc[500]{Computer systems organization~Embedded systems}
\ccsdesc[300]{Computer systems organization~Redundancy}
\ccsdesc{Computer systems organization~Robotics}
\ccsdesc[100]{Networks~Network reliability}

%%
%% Keywords. The author(s) should pick words that accurately describe
%% the work being presented. Separate the keywords with commas.
\keywords{datasets, neural networks, gaze detection, text tagging}

%% A "teaser" image appears between the author and affiliation
%% information and the body of the document, and typically spans the
%% page.
\begin{teaserfigure}
  \includegraphics[width=\textwidth]{img/sampleteaser}
  \caption{Seattle Mariners at Spring Training, 2010.}
  \Description{Enjoying the baseball game from the third-base
  seats. Ichiro Suzuki preparing to bat.}
  \label{fig:teaser}
\end{teaserfigure}

%%
%% This command processes the author and affiliation and title
%% information and builds the first part of the formatted document.
\maketitle

\section{Introduction}
There are many cases where a single language alone is not enough to meet the requirement of large and comlplex programs nowadays,
so using multiple languages is one of the promising way to implement such programs. For example ~~~

However, writing a relialbe multilingual program is not an easy task. (Too similar with ASE?)

Performing dataflow analysis of multilingual programs help developers to write more reliable code.

Many researchers have tried to perform static (dataflow?) analysis for multilingual programs.
(JN-SAF, Illia, ASE)

In this paper, we suggest a novel methodology for performing dataflow analysis of multilingual program.
(Which uses lightweight declarative style)

To evaluate the feasibility of our approach, we implemented the dataflow analysis for JNI programs using CodeQL.
(Which showed better performance)

We also tested out analyzer with real world applications retrieved from F-droid, and found new bugs and vulnerabilities from these apps.

The contribution of this paper is as follows:

+ We suggested a new methodology of using declarative style dataflow analysis for multilingual program

+ We implemented JNI dataflow analyzer with CodeQL, which outperforms state-of-art analyzers in terms of speed and precision.

+ We found and reported new bugs and vulnerabilities from the real-world JNI applications.

\section{Background}\label{sec:background} 
\begin{figure}[t]
  \centering
  \vspace{2mm}
  \includegraphics[width=0.94\textwidth]{img/ov1.pdf}
%  \vspace*{-1.5em}
  \caption{Overview of a declarative static analysis}
  \label{fig:ov1}
%\vspace*{-.5em}
\end{figure}

Figure~\ref{fig:ov1} presents the overview of how a declarative static
analysis works.  The analysis consists of three steps.  First, a given program
gets converted into syntactic facts. 
Second, new facts are generated by iteratively applying rules to the set of
known facts, until no new facts are derived.  
Finally, the query system takes a query and evaluates the rules with the given
facts, producing an analysis result for the query. 
The following paragraphs explain each step in detail, with examples written in
the Datalog-like syntax.

%Figure~\ref{fig:ov1} presents the overview of how a declarative style analysis
%works. The analysis consists of three steps.  First, a given program gets
%converted into database of syntactic facts.  Second, the rules that generate
%new facts are defined, and a declarative language engine evaluates the rules
%with the given facts, producing new facts.  Finally, the query is executed to
%extract all the facts that meet certain conditions, which correspond to the
%analysis result.

\textbf{Step 1: Extracting syntactic facts.}
The first step is to extract syntactic facts from a given program source.
Syntactic facts are the facts that can be statically determined
from the syntax of the given program.
For example, consider the following code:

\begin{lstlisting}[style=mcpp]
int f() {
  return 42;
}

int val = f();
\end{lstlisting}

For function definitions, the Syntactic Fact Extractor can extract syntactic
facts of a form \rcode{ Return(functionName, retExpr)}, where \rcode{Return}
denotes the relation between a string and a expression. \rcode{functionName} denotes the name of a
function, and \rcode{retExpr} denotes the return expression of the function.
Therefore, the extractor can extract the fact \rcode{ Return("f", 42)} from
the code.  
Another example syntactic fact is \rcode{ Call(callExpr, functionName)}, which
denotes that \rcode{ callExpr} is a call expression to a function named
\rcode{ functionName}.
For instatnce, the extractor can extract the following syntactic fact from the
code: \rcode{ Call(f(), "f")}.

%\inred{
%These syntactic facts serve as building blocks for the
%common Intermediate Representation (IR) of multiple languages.
%Compared to other IRs, this declarative-style IR has a few advantages.
%First, extracting information from source code in this format does not
%require any understanding of the language semantics, which imposes
%almost no performance overhead beyond parsing the source code.
%Second, the syntactic facts can be utilized easily in other kinds of
%analysis, since they are simple information that can be freely
%manipulated when defining new rules. Therefore, even when we use a different
%client analysis, we can reuse the extracted syntactic facts
%without re-extracting them from the source code.}

\smallskip
\textbf{Step 2: Deriving new facts by applying rules.}
The next step is to derive new facts from known facts by applying rules.
A rule defines a relation of that a fact, called {\it derivable fact}, can be
derived from a set of facts, called {\it base fact}. 
If all the base facts belong to known facts, the derivable fact is derived and
added to the known facts by the rule.
This process of deriving new facts is repeated, until no more new facts are
found.
%A rule indicates that if a certain facts are known, then a new rule can be
%generated and added to the set of known facts. This process of generating
%new facts is repeated, until no more new facts are found.


For example, consider a fact \rcode{ Step(x, y)} that denotes a direct dataflow
from a node \rcode{ x} to a node \rcode{ y}. 
Here, nodes represent program entities that can hold runtime values,
such as variables, literals, expressions, and function parameters. 
A direct dataflow can be established via the relation between a function call
expression and its callee function's return expression, because a function
call expression evaluates to its callee function's return value on runtime.
We can represent such dataflows as the following rule:
\begin{lstlisting}[style=mrule]
Step(retExpr, callExpr) :- Call(callExpr, functionName), Return(functionName, retExpr)
\end{lstlisting}

\noindent
\rcode{Step(retExpr, callExpr)} on the left-side of the rule is a derivable
fact, and \rcode{Call(callExpr, functionName)} and \rcode{Return(functionName,
retExpr)} on the right-side of the rule are base facts.  
This rule indicates that the new fact \rcode{Step(retExpr, callExpr)} is
derivable if the two base facts are known.
Since we already have \rcode{ Calls(f(), "f")} and \rcode{ Returns("f", 42) }
as known facts, the rule derives a new fact \rcode{ Step(42, f()) }.

Derived facts by rules are also known facts being used to derive other new
facts.
For example, we can think of another, \rcode{ Flow(x, z)}, that denotes a
dataflow from a node \rcode{ x} to a node \rcode{ z}.
We can define rules for the dataflow fact as the transitive closure of \rcode{
  Step}:

\begin{lstlisting}[style=mrule]
Flow(x, z) :- Step(x, z)
Flow(x, z) :- Step(x, y), Flow(y, z)
\end{lstlisting}

\noindent
The first rule indicates that the fact \rcode{ Flow(x, z)} is derivable if the
fact \rcode{ Step(x, z)} is known, and the second rule indicates
that the fact \rcode{ Flow(x, z)} is also derivable if the two facts \rcode{
  Step(x, y)} and \rcode{ Flow(y, z)} are known.
Thus, the rules derive \rcode{ Flow(42, f())}, \rcode{ Flow(f(), val)}, and
\rcode{ Flow(42, val)} from known facts, because \rcode{ Step(42, f())} and
\rcode{ Step(val, f())} are known.
%\inred{ \sout {
%The rules are usually
%evaluated in a bottom-up and modular manner, that is, each rule is evaluated
%one by one, after every rule it depends on is evaluated.
%}}

%The next step is to define rules to generate new facts out of known facts.
%%This step corresponds to actually implementing the algorithm of a
%%static analysis in a declarative style.
%For example, recall that we can define a call graph
%\rcode{ CallEdge(l1, l2)} using the facts
%named \rcode{ FunctionAt} and \rcode{ CallAt}.
%The defined rules are evaluated with declarative engines to finding all possible
%facts that can be derived. The rules are usually evaluated in a bottom-up
%and modular manner. Each rule is evaluated one-by-one, after every
%rule it depends on is evaluated. 

\smallskip
\textbf{Step 3: Performing queries.}
The final step is to perform the query via the query system.  
A query is a set of facts containing variables, and the query system finds
every variable assignment that makes the facts in the query belong to known
facts.
This step corresponds to actually obtaining the final result of a static
analysis in a declarative style.  
For example, one can make a specific query:

\begin{lstlisting}[style=mrule]
?- Flow(42, X)
\end{lstlisting}

\noindent
that queries all nodes into which the integer literal {\tt 42} flows.
When accepting the query as an input, the query system finds every value
\rcode{ v} such that \rcode{ Flow(42, v)} belongs to known facts generated in
the previous step. 
In this example, the query would give the query result
\rcode{ X\ $\in$ \{f(), val\}}.

%The final step is to perform queries via query system.
%The result of the query corresponds to actual result of client analysis.
%A query consists of set of facts, where some of the facts would have
%varaibles as arguments. Given a query, the query sytem will find all possible
%assignment on variables, that will make every facts would hold under
%the assignment.
%\newpage

\begin{figure}[t]
  \centering
  \vspace{2mm}
  \includegraphics[width=0.94\textwidth]{img/ov1.png}
%  \vspace*{-1.5em}
  \caption{Declarative analysis for monolingual programs}
  \label{fig:ov1}
%\vspace*{-.5em}
\end{figure}

%\section{Multilingual Program Analysis using Declarative Static Analyzers}
\section{Extension of declarative style analysis for multilingual programs}
\label{sec:approach}

In this section, we propose a general approach to extend a declarative
static analysis for multilingual programs.

\subsection{Declarative Analysis for a Single Language}

%A declarative analysis~\cite{doop} consists of facts and rules, and queries:
%\[
%  \begin{array}{llll}
%f & ::= & p(\overline{elem | x}) & \mbox{fact}\\
%r & ::= & f\ \mbox{:-}\ \overline{\neg^? f} & \mbox{rule}\\
%q & ::= & \mbox{?-} \overline{f}\ &  \mbox{query}
%\end{array}
%\]
%A fact ``$f = p(\overline{elem | x})$'' denotes a relation $p$
%between elements $\overline{elem}$ or variables $\overline{x}$.
%A rule ``$r = f'\ \mbox{:-}\ \overline{\neg^? f}$'' denotes that
%a fact $f'$ can be derived from other facts $\overline{f}$;
%$f'$ holds if all the facts $d \in \{\overline{\neg^? f}\}$ hold
%and all the facts $\neg d \in \{\overline{\neg^? d}\}$ does not hold.
%A query ``$\mbox{?-} \overline{f}$'' outputs all possible mappings from
%variables to elements, such that it will make all facts in $\overline{f}$ hold.
%
%Figure~\ref{fig:ov1} presents the overview of how a
%declarative-style analysis works for monolingual programs.
%The analysis consists of three steps.
%First, a given program gets converted into database of syntactic facts.
%Second, the rules that generate new facts are defined, and
%a declarative language engine evaluates the rules with the given facts,
%producing new facts.
%Finally, the query is executed to extract all the facts that meet
%certain conditions, which correspond to the analysis result.

\medskip
\textbf{Step 1: Extracting syntactic facts.}
The first step is to extract syntactic facts from a given source code.
Syntactic facts include facts about certain AST nodes and
the parent-child relationship between nodes. For example, consider
the following code:

\begin{lstlisting}[style=cpp,xleftmargin=2.5em]
function f() {
  return 42;
}
\end{lstlisting}
We can define a syntactic fact of the form ``\datalog{functionAt(l, name)},''
where \datalog{l} denotes the line number and \datalog{name}
denotes a name of the function defined at line \datalog{l}.
Therefore, we can extract the fact \datalog{functionAt(1, "f")}/
Another example syntactic fact is ``\datalog{enclosingStmt(l1, l2, i)},''
which denotes that the function at line \datalog{l1} has the statement
at line \datalog{l2} as the \datalog{i}-th subexpression.
For example, we can extract the following syntactic facts:
\datalog{enclosingStmt(1, 2, 0)}.

In a sense, these syntactic facts serve as building blocks for the
common Intermediate Representation (IR) of multiple languages.
Compared to other IRs, this declarative-style IR has a few advantages.
First, extracting information from source code in this format does not
require any consideration of the language semantics, which imposes
almost no performance overhead beyond parsing the source code.
Second, the syntactic facts can be utilized easily in any other kind of
analysis, since they are simple information that can be freely
manipulated by defining new rules. Therefore, even when we use a different
client analysis, we can reuse the extracted syntactic facts
without re-extracting them from the source code.


\medskip
\textbf{Step 2: Defining rules.}
The next step is to define rules to generate new facts out of known facts.
%This step corresponds to actually implementing the algorithm of a
%static analysis in a declarative style.
For example, recall that we can define a call graph
\datalog{callEdge(l1, l2)} using the facts
named \datalog{functionAt} and \datalog{callAt}.
The defined rules are evaluated with declarative engines to finding all possible
facts that can be derived. The rules are usually evaluated in a bottom-up
and modular manner. Each rule is evaluated one-by-one, after every
rule it depends on is evaluated. 

\textbf{Step 3: Performing queries.}
The final step is to perform queries via query system.
The result of the query corresponds to actual result of client analysis.
A query consists of set of facts, where some of the facts would have
varaibles as arguments. Given a query, the query sytem will find all possible
assignment on variables, that will make every facts would hold under
the assignment.

\begin{figure}[t]
  \centering
  \vspace{2mm}
  \includegraphics[width=0.94\textwidth]{img/ov2.png}
  \caption{Declarative analysis for multilingual programs}
  \label{fig:ov2}
%\vspace*{-.5em}
\end{figure}

\subsection{\mbox{Declarative Analysis for Multiple Languages}}
Now, we explain how we can compose two declarative-style static analyzers
for monolingual programs to perform analysis on multilingual programs.

Figure~\ref{fig:ov2} illustrates how we support multilingual
analysis in a declarative style in the case of two languages as an
example. The declarative language engine now gets two sets of
syntactic facts extracted from different languages. In addition,
new language-interoperation rules are defined on top of the original
language-specific rules from different languages to take the
interoperation semantics into account. Then, the same query is performed
to get the actual analysis.

%Let us revisit the dataflow analysis example for multilingual programs
%written in the language A and the language B.
%First, we merge each set of rules from each language.
%For example, we merge the rules for the fact \datalog{step} from each language:
%\begin{lstlisting}[style=myDatalog,xleftmargin=2.5em]
%step(a, b) :- step_A(a, b)
%step(a, b) :- step_B(a, b)
%\end{lstlisting}
%To take inter-language dataflows into account, we should also specify
%how a data of one language can be passed to another language cross language
%boundaries.  For example, one can pass data in one language into
%another by a function call as shown below:
%
%\begin{lstlisting}[style=java,xleftmargin=2.5em]
%public void main() { // Language A
%  int x = SOURCE();
%  B::f(x);
%}
%void f(int param) {  // Language B
%  printf("\%d", param);
%}
%\end{lstlisting}
%The variable \javacode{x} in the language A is passed to the parameter
%of a function \javacode{f} in the language B.  To reflect such a data flow,
%we can define the fact \datalog{foreignArgParam(a, b, i)},
%which denotes that \datalog{a} is the \datalog{i}-th argument of a foreign
%function call to a function whose \datalog{i}-th parameter is \datalog{b}.
%From the above example, we can derive the fact \datalog{foreignArgParam(x, param, 0)}.
%Using this fact, we can add a new rule to derive the common fact
%\datalog{step} as follows:
%\begin{lstlisting}[style=myDatalog,xleftmargin=2.5em]
%step(a, b) :- foreignArgParam(a, b, i)
%\end{lstlisting}
%which enables dataflows via foreign function calls cross language boundaries.

\begin{figure}[t]
  \centering
  \vspace{2mm}
  \includegraphics[width=0.7\textwidth]{img/codeql.pdf}
  \caption{Overall structure of \ours for JNI programs}
  \label{fig:codeql}
\end{figure}

\section{\ours: Extension of CodeQL for multilingual program analysis}\label{sec:impl}
In this section, we present \ours, a prototype extension of
CodeQL~\cite{codeql} for multilingual program analysis.  \ours tracks dataflows
across language boundaries in two types of multilingual programs: 1) Java-C
programs interoperating via Java Native Interface(JNI)~\cite{jnispec} and 2)
Python-C programs interoperating via Python Extension Module~\cite{pyext}.  For
simplicity, we use the Java-C program analysis to explain the common parts and
emphasize the Python-C program analysis in Section~\ref{sec:merging2}.

\subsection{CodeQL}
CodeQL is a declarative static analysis engine that transforms source code into
a database of facts and performs analyses by evaluating queries written in the
declarative and object-oriented language called QL (Query Language). Using QL,
one can depict rules by defining \textit{ predicates} and \textit{ classes}. The following
QL code defines the \inblue{classical Datalog-style} unary predicate \codeql{isOneOrTwo}:

\begin{lstlisting}[style=codeql,xleftmargin=2.5em]
predicate isOneOrTwo(int n) {
  n = 1 or n = 2
}
\end{lstlisting}

\noindent
which states that the fact \codeql{isOneOrTwo(n)} is derivable from either
of the two facts, \codeql{n = 1} and \codeql{n = 2}.

\inblue{ QL allows syntactic definitions of predicates which, like C functions,
can have return types. These are desugared into classical predicates
by adding an additional argument named \codeql{result}.
For example, consider} the following predicate \codeql{addOne}:

\begin{lstlisting}[style=codeql,xleftmargin=2.5em]
int addOne(int n) {
  isOneOrTwo(n) and result = n + 1
}
\end{lstlisting}
\noindent
The predicate is a syntactic sugar equivalent to the following \inblue{classical} predicate:
\begin{lstlisting}[style=codeql,xleftmargin=2.5em]
predicate addOnePred(int n, int result) {
  isOneOrTwo(n) and result = n + 1
}
\end{lstlisting}
\noindent
\inblue{
Then, every use site of the original predicate can be rewritten to
use the desugared predicates.
For example, an equality formula \codeql{m = addOne(n)} is desugared to
to \codeql{addOnePred(n, m)}.

One charateristic feature of QL is that it supports syntax of class similar to object-oriented
programing languages like Java. A QL class should be extended from another type,
and should define a special predicate inside the class, called {\it
characteristic predicate}. The characteristic predicates
resemble the syntax of the traditional constructors in other object-oriented languages;
it has the same name to the class and does not have return type.
Also, a QL class allows defining member predicates, as if defineing the methods of class.
These kinds of syntax allows the user to write intuitive QL codes in objected-oriented manner.
}
The following QL code defines a class named \codeql{OneOrTwo} \inblue{that extends \codeql{int}
with the characterisct predicate and one member predicate.}
:
\begin{lstlisting}[style=codeql,xleftmargin=2.5em]
class OneOrTwo extends int {
  // characteristic predicate
  OneOrTwo() { this = 1 or this = 2 }
  // member predicate
  int add(OneOrTwo that) { result = this + that }
}
\end{lstlisting}
\noindent

\inblue{
Despite its syntax, QL class is also a syntactic sugar of classical
predicates.  For example, the characterstic predicate \codeql{OneOrTwo} can be
desugared into a unary predicate, \codeql{isOneOrTwo} Note that the type of the
argument \codeql{this} is \codeql{int}, which the class \codeql{OneOrTwo} is
extending.
}
\begin{lstlisting}[style=codeql,xleftmargin=2.5em]
predicate isOneOrTwo(int this) {
  this = 1 or this = 2
}
\end{lstlisting}
\inblue{
\noindent
Similarly, The member predicate \codeql{add} can be desugared into a ternary predicate
by introducing two more arguments, \codeql{this} and \codeql{result}.
}
\begin{lstlisting}[style=codeql,xleftmargin=2.5em]
predicate addPred(OneOrTwo this, OneOrTwo that, int result) {
  result = this + that
}
\end{lstlisting}
\inblue{
Then, every use site of the class can also be transforemd to use the desugared predicates.
For example, \codeql{z = x.add(y)} where x has a type of class \codeql{OneOrTwo}
can be desugared into \codeql{isOneOrTwo(x) and addPred(x, y, z)}.
}
For more detailed information about CodeQL \inblue{and its language QL}, refer to the paper of Avgustinov et
al.\cite{ql2016} or the official document~\cite{codeql}.

Figure~\ref{fig:codeql} presents the overall structure of \ours for JNI
programs.  First, it generates databases for both languages, C\footnote{ Even
though \ours analyzes JNI programs written in Java and both C and C++, this
paper refers to C only for presentation brevity.} and Java, and merges them
into one database.  This corresponds to the step of extracting the initial
syntactic facts from the source code.  Then, \ours~merges the common rules in
both languages, which are parts of their libraries that CodeQL provides, into
one merged library.  Finally, using the merged database and merged library, a
user can write a query to perform a client-analysis and evaluate it to produce
its analysis result.

\subsection{Creating Databases}
For compiled languages such as C and Java, CodeQL generates their databases by
compiling source programs.  When a compiler compiles a program, CodeQL monitors
the compiler to extract necessary information and creates a database with the
extracted information. For scripting languages like Python, CodeQL uses its own
extractor to directly extract necessary information from source code.  CodeQL
creates a database for a single language in two steps: 1) it stores the
extracted information in a \inblue{single} \textit{trap} file, a human-readable format file for
the CodeQL database, and 2) it transforms the trap file into a database in
binary format.  For example, the following demonstrates a sample trap file:

\begin{lstlisting}[style=java,numbers=none]
#10001=@"class;myClass.MyClass"
#10002=@"type;int"
primitives(#10002,"int")
#10003=@"callable;{#10001}.myMethod({#10002}){#10002}"
#10004=@"params;{#10003};0"
params(#10004,#10002,0,#10003,#10004)
paramName(#10004,"myParam")
...
\end{lstlisting}

\noindent
which describes the parameter information of the method \javacode{myMethod}.
It also shows facts that will eventually be stored in tables:
\javacode{primitives(...)}, \javacode{params(...)}, and
\javacode{paramName(...)}.  The database manages tables based on fact types.
For example, it stores the fact \javacode{primitives(...)} in a table named
\javacode{@primitives}, and \javacode{params(...)} in a different table named
\javacode{@params}.

To create a single merged database for both C and Java, \ours maintains a
separate trap file for each language and then merges them into a trap file.
One problem is that if each trap file has a table with the same name, the merge
fails due to the name collision.  To avoid the problem, \ours renames each
table to a globally unique name by appending a language-specific prefix to the
table name before the merge.  For example, if a table named \codeql{@params}
exists in both C and Java trap files, \ours renames the table in C to
\codeql{@c\_params} and the table in Java to \codeql{@java\_params}.  After
renaming such tables, \ours merges the trap files into a database on which
predicates and queries are evaluated.


\subsection{Lifting Libraries}

CodeQL provides various libraries for C and Java, including pre-defined
predicates and classes for users to implement their own analyses.  A dataflow
analysis library is such a library, which supports both C and Java.  For
example, Figure~\ref{fig:qll} (a) and (b) show sample QL libraries for C and
Java, respectively, which use the same class name \ccode{Node} and the same
predicate name \ccode{localFlowStep}.  However, even with the same name, the
class \ccode{Node} in C and the class \javacode{Node} in Java are different
classes, which are incompatible.  The same applies to the predicate
\ccode{localFlowStep}.  In other words, we can not use the class \ccode{Node}
in C as an argument of the predicate \javacode{localFlowStep} in Java or vice
versa.

\begin{figure}[hbt!]
  \centering
%  \vspace{2mm}
  \begin{subfigure}{0.94\textwidth}
\begin{lstlisting}[style=codeql,xleftmargin=2.5em]
class Node { ... }

predicate localFlowStep(Node from, Node to) {
  // Expr -> Expr
  exprToExprStep_nocfg(from.asExpr(), to.asExpr())
  or
  // Assignment -> LValue post-update node
  ...
}
\end{lstlisting}
    \vspace*{-.5em}
    \caption{\normalsize c/dataflow/internal/DataFlowUtil.qll}
    \label{fig:qll1}
  \end{subfigure}
  \begin{subfigure}{0.94\textwidth}
\begin{lstlisting}[style=codeql,xleftmargin=2.5em]
class Node { ... }

predicate localFlowStep(Node node1, Node node2) {
  // Variable flow steps through assignment expression
  node2.asExpr().(AssignExpr).getSource() = node1.asExpr()
  or
  // Variable flow steps through adjacent def-use and use-use pairs.
  ...
}
\end{lstlisting}
    \vspace*{-.5em}
    \caption{\normalsize java/dataflow/internal/DataFlowUtil.qll}
    \label{fig:qll2}
  \end{subfigure}
  \begin{subfigure}{0.94\textwidth}
\begin{lstlisting}[style=codeql,xleftmargin=2.5em]
module C { /* original classes and predicates from C lib */ }
module JAVA { /* original classes and predicates from Java lib */ }

private newtype TNode =
  TJavaNode(JAVA::Node n)
  or
  TCNode(C::Node n)
class Node extends TNode {
  JAVA::Node asJavaNode() { this = TJavaNode(result)}
  C::Node asCNode() { this = TCNode(result) } ...
}

predicate localFlowStep(Node node1, Node node2) {
  JAVA::localFlowStep(node1.asJavaNode(), node2.asJavaNode())
  or
  C::localFlowStep(node1.asCNode(), node2.asCNode())
}
\end{lstlisting}
    \vspace*{-.5em}
    \caption{\normalsize jni/dataflow/internal/DataFlowUtil.qll}
    \label{fig:qll3}
  \end{subfigure}
  \vspace*{-.5em}
  \caption{QL libraries for C and Java}
  \label{fig:qll}
\end{figure}

To make classes and predicates in C and Java compatible, \ours lifts each
library to the common level.  First, as on lines 1 and 2 of
Figure~\ref{fig:qll3}, \ours encapsulates the original dataflow libraries for C
and Java in separate modules named {\tt C} and {\tt Java},
respectively\footnote{https://codeql.github.com/docs/ql-language-reference/modules/}.
After the encapsulation, the original classes and predicates can be referenced
via the module containing them.  For example, \codeql{Java::Node} refers to the
original class \codeql{Node} in the module {\tt Java}.  Lines 4 to 11 of
Figure~\ref{fig:qll3} demonstrate the lifted class of the original class {\tt
Node}.  \ours lifts a class by first defining a sum
type\footnote{https://codeql.github.com/docs/ql-language-reference/types/\#algebraic-datatypes},
denoting that the lifted class comes either from C or Java, and then making the
lifted class of that type.  The lifted class defines two member predicates,
\codeql{asCNode} and \codeql{asJavaNode}, that downcast the elements of the
lifted class into the elements of the original C or Java class.  Lines 13 to 17
of Figure~\ref{fig:qll3} demonstrate the lifted predicate of the original
predicate {\tt localFlowStep}.  Similarly, \ours lifts a predicate by combining
two original predicates with the \codeql{or} connective. For each predicate,
arguments and return values are downcasted to the elements of the class of
their corresponding language.  After lifting, lifted predicates show the
equivalent behaviors as the original ones if all the arguments are from the
same language.

\ours can fully automate the lifting process using QL compiler error messages.
When compiling a query without importing the library, the QL compiler reports
error messages containing required classes and predicates, and signatures of
the predicates. Using the information, \ours automatically synthesizes lifted
classes and predicates.

\subsection{Merging Libraries: Java-C}\label{sec:merging}
After lifting libraries for different languages, we manually extend predicates
to reflect the interoperation semantics between multiple languages.  For the
Java-C program analysis, we identified various interactions from Java to C and
vice versa via JNI, and extended predicates to model their behaviors.  For
example, the following shows how we extend the \inblue{CodeQL} predicate
named \codeql{viableCallable}:
\begin{lstlisting}[style=codeql,xleftmargin=2.5em]
DataFlowCallable viableCallable(DataFlowCall c) {
  result.asJavaNode()    = JAVA::viableCallable(c.asJavaNode())
  or result.asCNode()    = C::viableCallable(c.asCNode())
  or result.asCNode()    = viableCallableJ2C(c.asJavaNode())
  or result.asJavaNode() = viableCallableC2J(c.asCNode())
}
\end{lstlisting}

\noindent
It finds call edges from call expressions to their targets.  Lines 2 and 3 show
the results of lifting using original predicates from the dataflow libraries.
They handle intra-language call edges from Java to Java and from C to C,
respectively.  Lines 4 and 5 show the results of merging libraries,
representing inter-language call edges.  The predicate
\codeql{viableCallableJ2C} finds call edges from Java to C, and
\codeql{viableCallableC2J} finds call edges from C to Java.

\textbf{Java to C.} In Java-C programs, one can make interactions from Java to
C by calling native functions in C from Java code.  The target of such a
function call is determined in a static manner.  The target function should
follow the JNI naming convention, which is adding \codeql{Java\_} as prefix,
followed by a fully qualified name of its class and the additional \codeql{\_},
to the method name.  For example, the target function name for a function call
of \codeql{cfunction} would be
\codeql{Java\_fully\_qualified\_class\_name\_cfunction}.  With this convention,
we can define \codeql{viableCallableJ2C} so that \codeql{f =
viableCallableJ2C(call)} holds when \codeql{f.toString() = "Java\_" +
call.getTarget().className() + "\_" +} \\ \codeql{call.getTarget().getName()}
holds.

\textbf{C to Java.} The interaction from C to Java is more complex and requires
more careful implementation than the interaction from Java to C.  The primary
difference is that a method call from C to Java requires the runtime values of
variables, which may not be always possible.  First, C code calls the interface
function \ccode{GetMethodID(name, sig)} to get the ``method ID'' of the Java
method whose name matches the first argument and the type signature matches the
second argument passed to this function.  This method ID is stored at a
variable, say \ccode{mid}, and an actual method call is invoked by another
interface function, \ccode{Call<type>Method(obj, mid, args...)}. Calling this
interface function corresponds to calling the method that \ccode{mid} indicates
with \ccode{obj} as ``this object'' and \ccode{args} as the arguments.

To correctly handle this method call, we should be able to answer these
questions: ``When we call \ccode{GetMethodID}, what are the string values of
\ccode{name} and \ccode{sig}?'' and ``When we call \ccode{Call<type>Method},
what is the method ID value of \ccode{mid}?'' Soundly answering these questions
requires inter-language dataflow analysis because the string or method ID
values may be passed across language boundaries.  However, we observed that
such a pattern rarely happens in practice, and using only intra-language
dataflow analysis within C code is enough for most cases.  Therefore, we
decided to sacrifice the \inblue{soundness} by using intra-language analysis instead of
inter-language analysis.
\inblue{
This may result in missing some call edges in rare cases,
}
as a trade-off for a more lightweight and simpler
implementation.  We implemented two intra-language flow analysis modules for C,
which find 1) dataflows from string literals to the arguments of interface
functions and 2) dataflows from interface function call results to the
arguments of interface functions.  Using these modules, we can implement the
predicate \codeql{viableCallableC2J} by adding a call edge from a
\ccode{Call<type>Method} call to the method \ccode{m}, if there is a flow to
\ccode{mid} from a call to \ccode{GetMethodID}, and string values that flow
into the arguments \ccode{name} and \ccode{sig} of \ccode{GetMethodID} that
correspond to the name and the type signature of the method \ccode{m}.

In addition to the predicate \codeql{viableCallable}, we extended more
predicates to consider other JNI interface functions such as \ccode{findClass}
and \ccode{GetFieldID}.  Most of such extended predicates are specialized
\codeql{step} predicates.  We extended them in a similar way to the calls from
C to Java described above.

\subsection{Merging Libraries: Python-C}\label{sec:merging2}
To analyze Python-C programs, we extended predicates to model the
interoperation semantics of Python Extension Module.  The following shows
\codeql{viableCallable} predicate we extended for the Python-C program
analysis:

\begin{lstlisting}[style=codeql,xleftmargin=-.5em,numbers=none]
DataFlowCallable viableCallable(DataFlowCall c) {
  result.asPythonNode()    = PYTHON::viableCallable(c.asPythonNode())
  or result.asCNode()      = C::viableCallable(c.asCNode())
  or result.asCNode()      = viableCallableP2C(c.asPythonNode())
  or result.asPythonNode() = viableCallableC2P(c.asCNode())
}
\end{lstlisting}

\noindent
The only different parts from the Java-C program analysis are the predicates
\codeql{viableCallableP2C} and \codeql{viableCallableC2P}.

\textbf{Python to C.} Similar to the interactions from Java to C, one can make
interactions from Python to C by importing and calling functions from C. Python
Extension Module provides a pre-defined C struct \ccode{PyMethodDef} to export
a C function to Python:

\begin{lstlisting}[style=mcpp]
struct PyMethodDef methods[] = {
  {
    .ml_name = "cfunction",
    .ml_meth = cfunction_impl, ...
  }, ...
}
\end{lstlisting}

\noindent
The member field \ccode{ml_name} has a visible name of a C function to Python,
and the member field \ccode{ml_meth} has the pointer to the actual C function.
Thus, the function \ccode{cfunction_impl} defined in C code is invoked when
importing and calling \ccode{cfunction} in Python code.  To model the behavior,
we define the CodeQL class \codeql{PyMethodDef} and the predicate
\codeql{viableCallableP2C}. The class \codeql{PyMethodDef} corresponds to the C
structure \ccode{PyMethodDef}. It has two member predicates \codeql{getName()}
and \codeql{getFunc()} whose results are the values of the fields
\ccode{ml_name} and \ccode{ml_meth}, respectively.  Then, we make a rule for
\codeql{viableCallableP2C} such that for some \codeql{def} of type
\codeql{PyMethodDef}, if \codeql{def.getName() } \codeql{=
call.getTarget().toString()} for some \codeql{call}, then \codeql{def.getFunc()
} \codeql{= viableCallableP2C(call)} holds.

In addition, we define rules that connect dataflows from arguments of a
Python-to-C function call to parameters of its target C function.  Unlike the
Java-to-C function call semantics, Python Extension Module packs arguments in a
Python tuple object and propagates the object to a single parameter of the
target C function.  Then, the target C function unpacks the tuple object into
individual Python objects by calling the \ccode{PyArg_ParseTuple} API.  This
means that the usual way of connecting arguments and parameters via their
positions does not work, and we need another way.  We modelled this behavior by
defining \codeql{VirtualArgNode}, a subclass of \codeql{Node}, that corresponds
to the Python tuple object. Then, we define some predicates for representing
flows in and out of this node. First, we define a step that represents the flow
from values of the original argument nodes to virtual argument nodes. We then
define a step that represents the flow from the values of virtual argument
nodes to a single parameter node in a C function.  By doing so, the flows from
original argument nodes to use sites will be found in three steps: 1) from an
argument node to a virtual argument node, 2) from the virtual argument node to
the parameter node of a C function, and 3) from the parameter node to the out
node of \ccode{PyArg_ParseTuple} API function call.

\medskip

\textbf{C to Python.} The interaction from C to Python requires the runtime
values of variables, similar to the interaction from C to Java.  To invoke
Python functions in C, C code calls the interface function
\ccode{PyObject_CallObject(func, args)}, where \ccode{func} is a Python
function object, and \ccode{args} is a Python tuple object that contains all
arguments.  Because C code can get Python objects as function arguments of the
Python-to-C function calls, we define two ``intra-language flows'' analyses to
identify Python objects assigned to \ccode{func}: 1) dataflows from a Python
function object to the argument of a C function call, and 2) dataflows from the
parameters of a C function to the first argument of
\ccode{PyObject_CallObject}.  Using these intra-language flow analyses, we can
find actual targets that should be invoked for \ccode{PyObject_CallObject} API
function calls.

\section{Evaluation}
In this section, we show the feasibility of the approach of using declarative
analysis for multilingual program, and the effectiveness of its
implentation. More specifically, we set three research questions as follow:

RQ1) Feasibility: Is it possible to implement a working static analyzer for multilingual program in declarative style?

RQ2) Performance: How is the spped and precision of implemented analyzer, compared to the state-of-the-art analyzer?

RQ3) Usefulness: How practical is the implemented analyzer?

To answer RQ1, we ran our implemented analyzer to NativeFlowBench, which is a
small set of becnhmarks manually crafted by authors of JN-SAF[2], with the
purpose of testing dataflow analyzer for JNI programs. It consists of 23
android applications featuring various interactions between C++ and Java, some
of which contain a malicious code pattern that tries leak sensitive user data.
To answer RQ2, we ran our analyzer to 42 real-world android applications
collected from F-Droid[9], a repository for open-source android application. These
42 apps are selected by searching for apps with JNI, and among them, selecting those
that we could sucessfully compile. We compared our analyzer with
state-of-the-art[3] JNI analyzer, and confirmed that our analyzer outperforms
in terms of scalability and precision.  To answer RQ3, we implemented various
kinds of bug checkers, illustrated in previous researches[1][3], on top of our
analyzer. We ran our bug checker on the 42 applications of F-Droid mentioned
above, and discovered 38 bugs, including 30 newly found bugs.


\subsection{RQ1: Feasibility}
\includegraphics[width=0.5\textwidth]{img/table1}
The table1 shows the analysis result for 19 benchmarks in NativeFlowBench.  The
"Benchmark" coulmns denote the benchmark names, and "result" coulmns denote
whether the analysis result for corresponding benchmark was correct(O) or
not(X). We call that the analysis result is successful if every function call
target (call(j->c), call(c->j)) and field access tagret (field\_read(c->j),
field\_write(c->j)) are precisely determined, and every data leak is reported
correctly without false positives or false negatives.  The result shows that
except for one benchmark, our analyzer could correctly determine all targets
for function calls and field access correctly, and could successfully perform
dataflow analysis to find all of the data leaks. The only exception was
native\_compexdata\_stringop, where string manipulation functions such as
strcpy or strcat from C++'s standard library were used to create the string
value that indicates the target method's name, and since the inner-flow
analysis could not properly handle these functions, analyzer could not
correctly deterimine the function call target.

(Emphasize that data leak was not found in the previous research? or not?)

\subsection{RQ2: Performance}
\includegraphics[width=0.5\textwidth]{img/table2}
The table2 shows the analysis result for 42 F-Droid applications. The "time"
coulmn denotes the time for creating database and evaluating query. The average
time for creating DB was ??? seconds, and average time for evaluating query was
??? seconds, making the total analysis time ??? seconds. This is x4 faster on
average, compared to the state-of-the art JNI program analyzer.  Note that time
for creating DB includes compile time, and once DB is created, multiple queries
can be evaluated withou creating DB again. This means that pratically, the
pratical speed up becomes x14.

The "precision" coulms denotes the precision of dispatching function call
target, and field access target.  "Precise" means exactly one target was found,
and resolved means at least one target was found.  Compared to the previous
result, the precision was also higher for both function call targets and field
access targets.

\subsection{RQ3: Usefulness}


\section{Related Work}\label{sec:related}
%We propose the declarative-style static analysis of multilingual programs and
%actualize our approach as the JNI dataflow analyzer.
We describe three closely related categories of work below.
\subsection{Analysis for JNI}
Among various language representations of JNI programs, studies dealing with
pairs of languages are most closed related.  For JNI programs consisting of
Java and C source codes, ILEA~\cite{ILEA} extends the Java Virtual Machine
Language (JVML) and Jlint\footnote{\url{http://artho.com/jlint/manual.html}},
which is a JVML dataflow analyzer, to integrate C codes into one language.
Since a modest extension of JVML cannot fully support C semantics like memory
models, it extremely over-approximates C operations such as read and write
through pointers.  Lee et al.~\cite{LeeASE20} proposed a general approach to
analyze multilingual programs in the form of host and guest languages.  Their
approach first analyzes a guest language (C, in their implementation) to extract
semantic summaries (by Infer\footnote{\url{https://fbinfer.com/}}), translates
and integrates the summaries into a host language (Java), and then utilizes host
language analyzer (Flowdroid~\cite{Flowdroid}).  While their approach can
leverage the full features of existing static analyzers, ours inherits the
query-based analysis and can compute target properties effectively without
expensive and redundant summary generation.

Other studies deal with binaries rather than C/C++.
Fourtounis et al.~\cite{scanning} proposed a lightweight reverse engineering
technique to recover Java method calls in binary codes instead of performing
heavy analyses on binary code.  Their reverse engineering generates the data
facts of the Java calls with declarative style analyses for Java.
Their reverse engineering is lightweight yet targeting on Java method calls in
binary, in contrast, our approach targets on two general declarative style
analyses for each language and their interoperations seamlessly.
JN-SAF~\cite{JN-SAF} defines a unified summary in order to handle Java bytecode
and binary at once.  It extracts the summaries for each method with a Java
bytecode dataflow analyzer and a binary symbolic execution and composes the
summaries in a bottom-up manner to detect security vulnerabilities.
It sacrifices propagating dataflow facts until a fixed point by breaking a cycle
in the call graph arbitrarily to mitigate expensive computations from the
dataflow analysis and symbolic execution.  Our approach is scalable for the JNI
dataflow analysis even without sacrificing fixed point computations when C
source code is available.

\inred{Security analysis, dynamic analysis}

\subsection{Analysis for the other FFIs}
Android applications have another FFI between Java and JavaScript to maximize
their portability.  Android hybrid apps use JavaScript code to be run on
multiple browsers and can communicate with native Java code.
HybriDroid~\cite{HybriDroid} analyzes Java and JavaScript at the same time on
the top of the WALA framework~\cite{WALA} to detect programmer errors on interlanguage
communications and to track private data leakages.  Bae et al.~\cite{BaeICSE19}
tackled the expensive analysis of HybriDroid and proposed a light-weight type
system covering types of bug in HybriDroid.  However, their type system does not
cover tracking flows of private data.  Jin et al.~\cite{jin2014code} proposed a
static detection of code injection attacks from JavaScript to Java.  To avoid
expensive whole multilingual program analyses, they manually model APIs in the
PhoneGap\footnote{\url{https://cordova.apache.org/}} framework and apply a taint
analysis on JavaScript only.

\inred{Security analysis, dynamic analysis}

Python supports the interoperability mechanism with C~\cite{PythonC}.
Developers often import performance-oriented C code to high-level Python code.
Very recent work~\cite{sas2021} proposed a Python-C analyzer by reusing each
analyzer in the Mopsa framework~\cite{Mopsa}.  Their target properties require
precise context-sensitive value analyses.  Our approach easily activates only
the required features of each analyzer for such target properties.

\inred{(type systems)[12, 13, 14] OCaml, Java, C}

\subsection{Declarative-style analysis}
Declarative-style analysis alleviates challenges in crafting a static program
analyzer and thus Datalog \inred{[ten papers]} has become a dominant
DSL for static analysis~\cite{scholz2016}.
Bravenboer and Smaragdakis proposed the \doop framework~\cite{doop} using Datalog
that showed dramatic speedups for the Java points-to analysis.  They explain that
the modularity and declarativeness of Datalog rules are the keys to improve
performance~\cite{doopWorkshop}.  In addition, researchers have improved
analysis backends, declarative language engines%~\cite{bddbddb[37], µZ[16], Soufflé, madsen2016}.

Avgustinov et al.~\cite{ql2016} proposed QL, a declarative and object-oriented
query language to be compiled into Datalog and runs on a relational DB, and static
analyzers implemented QL which scale to millions of lines of code.
Our approach successfully inherits the fast and easy analysis of declarative-style
and emphasizes the benefits with the problem of multilingual program analysis.

\section{Conclusion}\label{sec:conclude}
Declarative static analysis has become a widely-used analysis technique but
has not supported multilingual programs actively developed in diverse
application domains.
In this paper, we present a practical extension methodology for a declarative static
analyzer that supports multiple languages to analyze multilingual programs.
The first step is to create a merged database consisting of multiple logical language
spaces. Each language space stores facts transformed from source code written in its
corresponding language.
Then, the second step is to define language-interoperation rules to derive facts
across language boundaries.
Our prototype implementation, \ours built on top of CodeQL, successfully tracks dataflows
over language boundaries in both Java-C and Python-C programs.
Using \ours, we found 33 true bugs and vulnerabilities from real-world
JNI applications, 12 of which are from the applications that \lees,
the state-of-the-art Java-C program analyzer, failed to analyze due to the lack
of the scalability.
We believe that our approach is applicable to various multilingual programs,
beyond Java-C and Python-C, and to multiple types of analyses.

%In this paper, we suggested the methodology of extending a declarative style
%anlyzer that targets multiple languages into a analyzer that can analyze the
%multilingual programs. First we create the database of fact that can be divided
%into multiple logical language spaces, where each language space consists of
%original facts. Then we define additional language-interoperation rules that
%will generate the facts that reflect interoperation semantics between languages.
%Finally, we perform the original query to obtain the analysis result.
%Based on this approach, we build our prototype implementation \ours on top of
%CodeQL.  Using \ours, we could sucessfully perform dataflow analysis on various
%benchmark suites of JNI programs and extension module programs.  Also, we find
%33 true bugs and vulnerabilities from real-world applications, 12 of which are
%from applications that \lees failed to detect.  We believe that our approach is
%applicable to analysis for other kinds of multilingual programs beyond Java-C
%or Python-C, and to various types of analysi beyond dataflow analysis.


%%
%% The acknowledgments section is defined using the "acks" environment
%% (and NOT an unnumbered section). This ensures the proper
%% identification of the section in the article metadata, and the
%% consistent spelling of the heading.
\begin{acks}
ACK. SYN. SYNACK.
\end{acks}

%%
%% The next two lines define the bibliography style to be used, and
%% the bibliography file.
\bibliographystyle{ACM-Reference-Format}
\bibliography{ref}

\end{document}
\endinput
%%
%% End of file `main.tex'.

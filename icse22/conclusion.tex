\section{Conclusion}\label{sec:conclude}
Multilingual programs are often vulnerable to bugs or security issues
more than monolingual programs due to the language interoperation semantics
via FFIs.  To prevent the issues, existing approaches leverages general-purpose
analyzers for monolingual programs.  Plenty of features in general-purpose
analyzers are harmful for scalability when the features are irrelevant to target
analysis properties.  Our approach extracts syntactic datafacts from the
parts written in different languages separately and defines additional rules to
describe the interaction between different languages.  Ours can effectively
check target properties in multilingual, thanks to the query-based-analysis
feature from a declarative-style analysis.  We implement the prototype
declarative-style JNI program analyzer on top of CodeQL.  Our analyzer showed
better scalability without analysis precision degradation compared to the
state-of-the-art JNI analyzer, \lees.  Moreover, we could easily develop
various kinds of bug detectors on top of our analyzer.  Using these bug detectors,
we found 33 true bugs and vulnerabilities from real-world applications, and 12 out
of 33 bugs are from the applications that \lees failed to analyze.  We believe
that our approach is applicable to various FFIs and can prevent interoperation
bugs and vulnerabilities in multilingual programs.

\section{General Approach}

In this section, we propose a gneralized approach for performing declarative
dataflow analysis for a multilingual program.

\subsection{Declarative dataflow analysis}

We start by defining a declarative style dataflow analysis for one language.
First, program points that can hold values are represented as nodes. Example of
nodes would be varaibles, literals, parameter, arguments, or object fields.
Now, we can express dataflow analysis in declarative style as flow(a, b), which
means that value of the node a can flow into node b. flow(a, b) can be defined
as transitive closure of a predicate step(a, b);

flow(a,b) :- a = b or step(a,c) and flow(c, b)

The predicate step(a,b) means that there is a direct flow from node a to b. Assuming that
out target languages have function call and field access, then step (a,b) would be
defined as following:

step(a,b) :- localStep(a, b) or functionInStep(a, b) or functionOutStep(a, b)
or filedReadStep(a, b) or fieldWriteStep(a, b)

Here, localStep represents any flows that do not envolve function call or field
access.  Example would be assigning to a variable, or simple def-use pattern.
functionInStep and functionOutStep are used for flows envolving function calls,
such as flow from argument to parameter or flows regarding return statement.
fieldReadStep and fieldWriteStep are predicates for field access.

Two predicates, functionInStep and functionOutStep, can be defined using more basic predicates:

functionInStep(arg, param) :- isArgument(call, arg, i) and isParameter(function, param, i) and callEdge(call, function)
functionOutStep(ret, pos) :- return(function, ret) and returnPosition(call, pos) and calLEdge(call, function)

The predicate isArgument(call, arg, i) means that arg is the ith argument of
given call, and similarly, the predicate isParameter(function, param, i) means
that param is the ith parameter of given function.  The predicate
return(function, ret) indicates that there is a return statement returning the
value ret in given function, and the prediacate returnPosition(call, pos)
indicates that the returned value of given call would be stored in the node
ret.  Finally, the callEdge(call, function) means that the given call's target
is the given function.

With the all of above rules defined, once we provide facts for predicates that
do not appear as head position for rules (a.k.a. external predicates) such as localStep, fieldReadStep, or
callEdge, then the system can successfully find all pair of nodes (a,b) that
satisfies flow(a,b).

TODO: exmaple with simple language?

\subsection{Extension to dataflow analysis of multilingual language}

Note that providing facts for external predicates was required to complete
dataflow analysis of single-language program.  This is also true for
multi-lingual language. Let's assume that we are given all required facts for
each of two languages separately. For predicates that are not envolved with interaction
between languages, all we have to do is just merging those predicates together. For example,
the facts for localStep for multilingual analysis would be simply the union of facts for localStep
of each language. For predicates that are envloved with interaction, they should be extended further,
considering the semantics of the operation. Such predicates would include 3 predicates:
callEdge(call, fun), fieldReadStep(a,b) ,fieldWriteStep(a,b). Let's take a callEdge predicate as an
example. By simply merging existing two predicates, then for every pair taht satisfies callEdge(call, fun),
both call and fun would be from either the first language or the second language. In addition to that,
we must find and add all pair of call and fun, where call is from the first language and the fun from the second,
or vice versa.

For static interactions, such extensions are not difficult, since new facts can be easily recognized.
For dynamic interactions however, finding such facts are not trivial, since in order to determine the
target, the runtime value is needed. In this case, one can perform "inner dataflow analysis", only targeting
one language. For example, when a dynamic function call of form callMethod(funnName, arg) is made,
then one can find all possible string literal values that can flow into the argument "funName". Using this
analysis result, the function target is determined, and then the new fact for predicate callEdge can be added.

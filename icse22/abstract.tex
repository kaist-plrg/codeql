\begin{abstract}
The use of multilingual programs are growing, and the demand for the static
analysis of multilingual programs is also growing correspondingly.  Since most
existing static analyzers target one language, the static analysis of
multilingual programs are done by either translating one language into another
and applying static analyzer for the translated language, or combining two
analyzers so that the two analyzer give output in unified format, and analysis
is done with respect to the output.  In whichever approach, the problem
is that the system consists of too many components which treat each others
as blackbox, resulting in lacks of flexibility and extensibility.

In this paper, we suggest a novel methodology for implementing static analysis
for multilingual program, which is using declarive style analysis, for the
first time. In declarive style analysis, everything is expressed in terms of
"data-facts" and "rules". The analysis result and the information required for
calculating it is expressed in "data-facts", and how one can derive new
data-facts from the known data-facts.  This simple structure makes declraritive
stlye analysis more flexible and extensibie, alleviating the innate problem of
multilingual analysis.

To show the feasibility of our approach, we implementated dataflow analyzer for
JNI programs, using CodeQL, and compared its performance in terms of
scalibility and precision with a state-of-the-art JNI analyzer. The experiment
result shows that our implementation is x14 faster, and has ??\%p better
precision. In addition, we implented bug checkers on top of our analyzer, and
we could find bugs in real-world JNI applications.
\end{abstract}

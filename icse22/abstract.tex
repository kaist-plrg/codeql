\begin{abstract}
The use of multilingual programs are growing, and the demand for the static
analysis of multilingual programs is also growing correspondingly.  Since most
existing static analyzers target one language, the static analysis of
multilingual programs are done by either translating one language into another
and applying static analyzer for the translated language, or combining two
analyzers so that the two analyzer give output in unified format, and analysis
is done with respect to the output.  In whichever approach, the problem
is that the system consists of too many components which treat each others
as blackbox, and each component communicates with each other through a fixed
form of API. This restriction results in lacks of flexibility and extensibility of
the implementation.

In this paper, we suggest a novel methodology for implementing static analysis
for multilingual program, which is using declarative style analysis, for the
first time. In declarative style analysis, everything is expressed in terms of
"data-facts" and "rules". The analysis result and the information required for
calculating it is expressed in "data-facts", and how one can derive new
data-facts from the known data-facts is expressed in "rules". This simple
structure makes declarative style analysis more flexible and extensible,
alleviating the innate problem of multilingual analysis.

To show the practicality of our approach, we implemented dataflow analyzer for
JNI programs, using CodeQL, and compared its performance in terms of
scalability and precision with a state-of-the-art JNI analyzer. The experiment
result shows that our implementation is x14 faster, and has ??\%p better
precision. In addition, we implemented bug checkers on top of our analyzer, and
we could find bugs in real-world JNI applications.
\end{abstract}

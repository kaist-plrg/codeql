\begin{abstract}
Multilingual programs written in multiple programming languages together have
been widely developed for various purposes, but they may have more bug and
vulnerability surfaces than monolingual programs. Previous research has
proposed static analysis techniques for multilingual programs utilizing
multiple static analyzers together. The approaches abstract behaviors of
whole modules written in a single language as summaries using an analyzer,
and use another analyzer that analyzes modules written in another language
with the summaries to detect bugs and security vulnerabilities cross language
boundaries. While they leverage full features of multiple static analyzers,
they suffer from scalability issues because the summary generations require
expensive static analyses even for modules unrelated to their target
properties. 


In this paper, we propose a simple and effective declarative style static
analysis for multilingual programs, for the first time. Our approach extracts
{\it datafacts} from modules written in each language respectively, and
applies {\it rules} ... (Require more precise high-level description for our
approach). Because our approach directly works on datafacts syntactically
extracted from modules and inherits the query-based-analysis feature from the
declarative style analysis, it can effectively check target properties in
multilingual programs without expensive and redundant summary generations.
As a proof-of-concept, we implement a JNI program static analyzer on top of
CodeQL, which detects bugs and security vulnerabilities by tracking dataflows
over language boundaries between Java and C. In our empirical evaluation, we
show that the analyzer is {\it precise} in that it analyzed dataflows with
higher precision, and {\it scalable} in that it detected the same types of
bugs and security vulnerabilities about 14x faster, than the state-of-the-art
analyzers.
\end{abstract}

%The use of multilingual programs are growing, and the demand for the static
%analysis of multilingual programs is also growing correspondingly.  Since most
%existing static analyzers target one language, the static analysis of
%multilingual programs are done by either translating one language into another
%and applying static analyzer for the translated language, or combining two
%analyzers so that the two analyzer give output in unified format, and analysis
%is done with respect to the output.  In whichever approach, the problem
%is that the system consists of too many components which treat each others
%as blackbox, and each component communicates with each other through a fixed
%form of API. This restriction results in lacks of flexibility and extensibility of
%the implementation.
%
%In this paper, we suggest a novel methodology for implementing static analysis
%for multilingual program, which is using declarative style analysis, for the
%first time. In declarative style analysis, everything is expressed in terms of
%"data-facts" and "rules". The analysis result and the information required for
%calculating it is expressed in "data-facts", and how one can derive new
%data-facts from the known data-facts is expressed in "rules". This simple
%structure makes declarative style analysis more flexible and extensible,
%alleviating the innate problem of multilingual analysis.
%
%To show the practicality of our approach, we implemented dataflow analyzer for
%JNI programs, using CodeQL, and compared its performance in terms of
%scalability and precision with a state-of-the-art JNI analyzer. The experiment
%result shows that our implementation is x14 faster, and has ??\%p better
%precision. In addition, we implemented bug checkers on top of our analyzer, and
%we could find bugs in real-world JNI applications.

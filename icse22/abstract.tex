\begin{abstract}
Multilingual programs are written in multiple programming languages
taking advantage of their own language features and benefits.
While multilingual programs have been widely developed for various purposes,
they are more vulnerable to bugs and security issues than monolingual programs.
Thus, researchers have proposed two approaches to analyze multilingual programs.
The first approach leverages existing static analyzers for
monolingual programs to analyze multilingual programs.
It first analyzes each part written in a single language
using its corresponding analyzer to generate its summary and then
perform \textit{a target analysis} to detect bugs and security vulnerabilities cross
language boundaries using the generated summaries.
The second approach abstracts the behaviors of
each part written in a single language into a single form
and performs a target analysis using existing top-down analyzers.
While they can utilize the full features of existing static analyzers,
they often fail to analyze large-scale real-world applications.
Because existing static analyzers support abstractions of all the language features,
they may waste their analysis time on the features that may not be
relevant to the target analysis. 

In this paper, we propose a simple and effective declarative-style static
analysis for multilingual programs for the first time.
A declarative-style analysis expresses analysis information as {\it datafacts},
describes how to derive a datafact from other datafacts as {\it rules},
and performs a {\it query-based} analysis considering only relevant datafacts.
Our approach extracts syntactic datafacts from the parts written in different
languages separately and defines additional rules to describe the interaction
between different languages.  Because our approach works directly on datafacts
extracted from programs and inherits the query-based-analysis
feature from a declarative-style analysis, it can effectively check target
properties in multilingual programs without irrelevant computations.
As a proof-of-concept, we implement a JNI program static analyzer \ours
on top of CodeQL, which detects bugs and security vulnerabilities by tracking
dataflows over language boundaries between Java and C.
Our evaluation shows that \ours is {\it precise} in that it analyzes
dataflows with higher precision and is {\it scalable} in that it analyzes
more applications than the state-of-the-art analyzers.
It also reported 33 genuine bugs in real-world JNI Android applications.
\end{abstract}

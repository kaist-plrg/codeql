\section{Motivation}
In this section, we show how multilanguage interactions work using a
simple example code, and explain challenges that are difficult to handle
by the previous approaches but our declarative-style analysis can handle.

\subsection{Example target multilingual program}

Let's consider two multilingual programs, written in (Java-like) language A
and (C-like) language B.
\begin{figure}[t]
  \centering
  \vspace{2mm}
  \begin{subfigure}[t]{0.5\textwidth}
    \begin{lstlisting}[style=java,xleftmargin=2.5em]
public void main() { //Language A
  Node node = null;
  B::f(node);
}
    \end{lstlisting}
    \vspace*{-.5em}
    \caption{Language A}
    \label{fig:exam1:langA}
  \end{subfigure}
  \begin{subfigure}[t]{0.5\textwidth}
    \begin{lstlisting}[style=cpp,firstnumber=5,xleftmargin=2.5em]
void f(Node* param) { //Language B
  if(param != NULL) {
    Node* ptr = param;
    ptr->next = new Node;
  }
}
    \end{lstlisting}
    \vspace*{-.5em}
    \caption{Language B}
    \label{fig:exam1:langB}
  \end{subfigure}
  \vspace*{-.5em}
  \caption{Prgram 1}
  \label{fig:exam1}
\end{figure}

In the first program(Figure \ref{fig:exam1}), value null from language A is
passed to the parameter of function language B via calling it. Then, in
language B, it is checked whether the parameter has null value or not, and only
when it is not null, the parameter is safely dereferenced.

\begin{figure}[t]
  \centering
  \vspace{2mm}
  \begin{lstlisting}[style=cpp,firstnumber=5,xleftmargin=2.5em]
void f(Node* param) { //Language B
  Node* ptr = param;
  Node* next;
  if(condition()) {
    next = NULL;
  } else {
    next = new Node;
  }
  ptr->next = next;
}
  \end{lstlisting}
  \vspace*{-.5em}
  \caption{Program 2}
  \label{fig:exam2}
\end{figure}

In the second program(Figure \ref{fig:exam2}), the code for language A is same
as before. The difference is in code for language B. This time, the code does
not check whether or not the parameter is null. Instead, there is another
if-statement, which is not directly related parameter, but affects value of
  another variable.

\subsection{Previous approaches for performing analysis on target program}

Let's assume that we want to perform dataflow analysis for these programs.
What we are interested is whether the value null from language A can be
dereferenced in language B or not. In the first program, the parameter is
dereferenced only if it is not null, so there is no such a bug. Note that in
order to correctly conclude that there is no error, the control flow should be
considered. On the other hand, in second program, since we do not check the
value of parameter, it should be reported that null dereference error might
happen.

In previous approaches[2][3], existing analyzers are utilized to generate
\textit{summaries}. Then the actual target-analysis, the dataflow analysis, is
performed on these summaries. The problem is that since such existing analyzers
are mostly general purpose, unnecessary calculations which are not needed for
target-analysis would happen during genrerating the summaries, resultingin the
performance overhead. For example, unlike the first program where calculating
information about if-statement and generating summary regarding it
would be required. calculating the information about if-statement in the second
program is not necessary for detecting the dataflow from null to ptr, yet there
is no easy way to "turn off" such heavy computation for this specific
if-statement.

\subsection{Declarative style multilingual program analysis}

By using declarative style analysis, we can solve the problem mentioned above.
The main reason is in its query-based analysis: the target-analysis is
written in terms of \textit{query}, and only related data-facts and rules
are used for evaluating the query, reducing the amount of computation.

Let's see how a dataflow analysis can be implemented in declarative style.
Let's define \datalog{node} to be the program entity that can hold the run-time
value, such as variable, literal, or function parameter.  Ultimately, the
result data-fact we want would be \datalog{flow(a,b)}, which will denote the
fact that there is a data flow from \datalog{node a} to \datalog{node b}. The
rule for this data-fact can be defined in terms of another data-fact named
\datalog{step}:

\begin{lstlisting}[style=myDatalog,xleftmargin=2.5em]
flow(a,b) :- step(a,b)
flow(a,b) :- step(a,c), flow(c,b)
\end{lstlisting}

Simply put, this rule states that \datalog{flow} is defined as transitive
closure of \datalog{step}.  Here, \datalog{step(a,b)} denotes the direct data
flow from \datalog{node a} to \datalog{node b}. Again, the data-fact
\datalog{step} can be calculated by rules using another data-facts, which will
be eventually broken down to basic syntactic data-facts.  Then, given the set
of all syntactic data-facts, all the other possible data-facts, including
\datalog{step} and \datalog{flow}, can be calculated. For example, for the
second program, the following data-facts will be generated,

\begin{lstlisting}[style=myDatalog,xleftmargin=2.5em]
step(null, list) // A -> A
step(list, arg) // A -> B
step(arg, ptr) // B -> B
step(ptr, [dereference of ptr]) // B -> B
\end{lstlisting}

then the declarative language engine can compute the result \datalog{flow(null,
[dereference of ptr])}, and the analyzer concludes that there is a dereference
of null. In process of deriving this conclusion, there was no need to consider
the if-statement.

This time, let's say we are analyzing the first program, which requires
considering the control flow.  One way to implement it would be to bring the
concept of guard. We can define the data-fact \datalog{guard(node, cond)} to
indicate that \datalog{node} is contained in the scope of if statement whose
condition is \datalog{cond}.  For example, in the first program, the
\datalog{ptr} node would be guarded with the condition, \ccode{param \!= NULL},
and this is expressed by the data-fact \datalog{guard(ptr, "param \!= NULL")}.
Once the guard is implemented, then all we have to do is to give additional
condition to the data-fact \datalog{step}, so that \datalog{step(a,b)} does not
hold if \datalog{a} or \datalog{b} is guarded by a proper guard, for example,
it contains comparison with null.

The important thing here is that, unlike previous approaches that considers
every if-statement in order to generate the summary, it is not required to
handle all the if-statements in our approach. We can selectively calculate the
certain data-facts regarding certain control flow, only the target-analysis
does require information about it.

In the following sections, we formalize the general approach for performing
declarative style analysis in multilingual program (Section 3), show the
specific implementation of this approach for JNI programs as form of dataflow
analysis, implemented with CodeQL(Section 4) and show the evaluation result of
this implementation (Section 5).

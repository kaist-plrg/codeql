\section{Motivation}
In this section, we show how multilanguage interactions work using a
simple example code, and explain the main challenge that is difficult to handle
by the previous approaches but our declarative-style analysis can handle.

\begin{figure}[t]
  \centering
  \vspace{2mm}
  \begin{subfigure}[t]{0.5\textwidth}
    \begin{lstlisting}[style=java,xleftmargin=2.5em]
public void main() {
  Node node = null;
  B::f(node);
}
    \end{lstlisting}
    \vspace*{-.5em}
    \caption{Method in language A calling a function in language B}
    \label{fig:exam1:langA}
  \end{subfigure}
  \begin{subfigure}[t]{0.5\textwidth}
    \begin{lstlisting}[style=cpp,firstnumber=5,xleftmargin=2.5em]
void f(Node* param) {
  if(param != NULL) {
    Node* ptr = param;
    ptr->next = new Node;
  }
}
    \end{lstlisting}
    \vspace*{-.5em}
    \caption{Function in language B receiving null from language A}
    \label{fig:exam1:langB}
  \end{subfigure}
  \vspace*{-.5em}
  \caption{Multilingual code in language A and language B}
  \label{fig:exam1}
\end{figure}

\begin{figure}[t]
  \centering
  \vspace{2mm}
  \begin{lstlisting}[style=cpp,firstnumber=5,xleftmargin=2.5em]
void f(Node* param) {
  Node* ptr = param;
  Node* next;
  if(condition()) {
    next = NULL;
  } else {
    next = new Node;
  }
  ptr->next = next;
}
  \end{lstlisting}
  \vspace*{-.5em}
  \caption{Code in language B with a conditional statement}
  \label{fig:exam2}
\end{figure}

\subsection{Interactions between Multilingual Modules}

Let us consider two multilingual programs written in a Java-like language A
and a C-like language B.

The first program in Figure~\ref{fig:exam1} has two parts:
Figure~\ref{fig:exam1}(a) shows a method \javacode{main} in language A,
and Figure~\ref{fig:exam1}(b) shows a function \ccode{f} in language B.
The method \javacode{main} calls the function \ccode{f} with the argument
\javacode{node} whose value is \javacode{null}, which passes the null
value from language A to language B. The function \ccode{f} then
checks whether the value of its parameter \ccode{param} is null
and dereferences it only when it is not null.

The second program has the same method in Figure~\ref{fig:exam2}(a)
and the function \ccode{f} in Figure~\ref{fig:exam2}.
Even though the value of \ccode{f}'s parameter \ccode{param} may be \javacode{null},
\ccode{f} does not check its value and simply dereferences it on line 13.
In addition, \ccode{f} has an extra conditional statement on lines 8--12,
which is irrelevant to \ccode{param}.

\subsection{Previous Approaches}
Now, let us perform a dataflow analysis on these programs to detect
the null dereference bug, that is, whether the null value from
language A can be dereferenced in language B or not.
In the first program, since the parameter of \ccode{f} is
dereferenced only when it is not null, it does not have the bug.
Note that to conclude the absence of the bug correctly,
the analysis should also consider control flows.
On the other hand, in the second program, since \ccode{f} does not
check the value of its parameter, it may have the null dereference bug.

The previous approaches either 1) utilize existing analyzers for
languages A and B to generate their summaries and perform the
target dataflow analysis~\cite{LeeASE20,JN-SAF}, or 2) abstract 
the behaviors of the code written in A and B into a single form and
perform the target analysis~\cite{hybridroid,sas2021}
The problem is that since such existing analyzers are mostly general-purpose,
which may have unnecessary computations that are not necessary for the target analysis,
these approaches may result in performance overhead.
For example, even though the analysis of the conditional statement in Figure~\ref{fig:exam2}
is not relevant to the detection of the null dereference bug on line 13,
the conventional target analysis in the previous approaches computes
the information about the conditional statement during its analysis.
This performance overhead would become critical in multilingual analysis
because the analysis of the interactions between different languages would require
the analysis of immense analysis information in both languages.


%\subsection{Declarative style multilingual program analysis}
\subsection{Our Approach}

Finally, our approach is based on a declarative-style analysis, which
alleviates the performance overhead problem of the previous approaches.
The main benefit of the approach comes from its query-based analysis:
a target analysis is
specified in terms of \textit{queries}, and only relevant datafacts and rules
are used for evaluating queries, reducing the amount of computation.

To illustrate, we first explain a declarative dataflow analysis of the
first program considering necessary control flows. Then, we explain
how the computation of control flows for the second program can be
reduced by a declarative analysis.

For the first program, the target analysis is to find \datalog{flow(a, b)},
which denotes that there is a dataflow from \datalog{node a} to \datalog{node b},
where \datalog{node} represents a program entity that can hold a run-time
value, such as a variable, a literal, and a function parameter.
We can define the rule for this datafact as the transitive closure of
the datafact named \datalog{step}:
\begin{lstlisting}[style=myDatalog,xleftmargin=2.5em]
flow(a, b) :- step(a, b)
flow(a, b) :- step(a, c), flow(c, b)
\end{lstlisting}
where \datalog{step(a, b)} denotes a direct data
flow from \datalog{node a} to \datalog{node b}.
Because \datalog{step} denotes both an intra-language flow and an inter-language flow,
the following datafacts are generated:
\begin{lstlisting}[style=myDatalog,xleftmargin=2.5em]
step(null, node)  // A -> A
step(node, param) // A -> B
\end{lstlisting}
Since the conditional statement on line 6 checks whether \ccode{param}
is null, the null value does not flow into the variable \datalog{ptr}.
Therefore, our rule should not generate the datafact \datalog{step(param, ptr)}.
One way to implement it is using ``guard'': the datafact \datalog{guard(node, cond)}
indicates that \datalog{node} is in the scope of the conditional
statement whose condition is \datalog{cond}.
In the first program, the \datalog{ptr} node is guarded with the condition
``\ccode{param \!= NULL},'' and we can express it as the datafact
\datalog{guard(ptr, "param \!= NULL")}.
Using this datafact \codeql{guard}, we can give an additional condition to the
datafact \datalog{step}. For example, we can express that
\datalog{step(a, b)} does not hold if \datalog{node b} is in the scope of
a conditional statement and is guarded by a proper condition like null checking.

Now, let us consider the second program. Using the same approach, we
can generate the following datafacts, including flows cross language boundaries:

\begin{lstlisting}[style=myDatalog,xleftmargin=2.5em]
step(null,  node)  // A -> A
step(node,  param) // A -> B
step(param, ptr)   // B -> B
step(ptr,   [dereference of ptr]) // B -> B
\end{lstlisting}
Then, the analysis can compute
\datalog{flow(null, [dereference of ptr])},
and the analyzer concludes that the second program has the null dereference bug.
Note that the trace of \datalog{step}s does not contain any nodes
inside the conditional statement.  Therefore, in the process of
detecting the bug, the analysis does not need to consider the
irrelevant nodes in the conditional statement.
Unlike previous approaches that consider every conditional statement
regardless of whether the analysis of the conditional statement is
necessary or not, our approach does not require analysis of irrelevant code.

In the remainder of this paper, we formalize the general approach of a
declarative static analysis for multilingual programs (Section~\ref{sec:approach}), show a
specific implementation of the approach for JNI programs in the form of dataflow
analysis, implemented in CodeQL (Section~\ref{sec:impl}), discuss the evaluation results of
the implementation (Section~\ref{sec:eval}) and related work (Section~\ref{sec:related}), and
conclude (Section~\ref{sec:conclude}).
